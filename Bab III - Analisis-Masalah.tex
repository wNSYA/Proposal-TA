% ============================================================================================
% BAB III ANALISIS MASALAH
% Pembagian subbab tidak rigid dan dapat bervariasi. Bab ini minimal berisi analisis kebutuhan
% fungsional dan nonfungsional, analisis berbagai alternatif solusi yang dapat ditawarkan, dan
% metode pemilihan solusi yang diusulkan.
% ============================================================================================
\chapter{ANALISIS MASALAH}
\label{chap:analisis-masalah}
\section{Analisis Kondisi Saat Ini (\textit{Emphatize})}
Berdasarkan hasil observasi lapangan dan wawancara terhadap pengurus lapangan di Gedung ITB Innovation Park, alur masuk pengunjung maupun tamu pada saat ini masih dilakukan secara manual dan belum teritegrasi. Alur kontrol akses yang ada saat ini adalah sebagai berikut:
\begin{enumerate}
    \item   Setiap orang yang hendak memasuki gedung akan ditanyakan oleh petugas pertanyaan berupa asal lembaga atau perusahaan beserta tujuan memasuki gedung. Petugas akan menilai dan memutuskan sendiri apakah pengunjung diperbolehkan masuk.
    % \item   Setelah diperbolehkan masuk, jika menggunakan kendaraan, pengunjung akan diarahkan menuju area parkir. Di tempat parkir, petugas akan mencatat nama beserta plat kendaraan dari pengunjung. jika tidak, pengunjung akan diarahkan langsung ke lobi.
    \item   Setelah diperbolehkan masuk, pengunjung akan diarahkan langsung ke lobi untuk bertemu resepsionis atau petugas yang sedang berjaga di lobi.
    \item   Resepsionis ataupun petugas keamanan yang berjaga kemudian akan menanyakan tujuan pengunjung, dan terkadang akan memandu pengunjung dalam menggunakan lift ke area tersebut.
\end{enumerate}
Berdasarkan alur, terlihat bahwa kondisi saat ini memiliki kelemahan yang penting, terutama pada proses dokumentasi informasi dari pengunjung gedung yang hanya dilakukan di area parkir. Selain itu, sistem saat ini memiliki kerentanan pada keputusan memberikan akses yang sepenuhnya dilakukan oleh petugas keamanan serta parameter pemberian akses yang subjektif. Selain itu, sistem kontrol akses saat ini belum terintegrasi dengan Sistem Manajemen Bangunan Gedung. Mengingat fungsi gedung sebagai pusat inovasi yang menyimpan aset bernilai tinggi, sistem saat ini tidak lagi layak untuk diterapkan dan membutuhkan perubahan baik infrastruktur maupun regulasinya. 

Selain alur kontrol akses yang masih manual, kondisi lobi pada Gedung IIP saat ini belum memiliki infrastruktur kontrol akses yang memadai, dengan hanya resepsionis dan petugas penjaga sebagai bagian dari kontrol akses manual. Gambar \ref{gambar:lobi-iip} menunjukkan gambaran kondisi saat ini dari lobi Gedung IIP.
\begin{figure}[H] % pilihan opsi yang disarankan: t = top, b = bottom, h = here
	\centering
  \captionsetup{justification=centering}
    	\includegraphics[width=0.5\textwidth]{image/denah-sebelum.png}
	\caption{Kondisi lapangan dari lobi Gedung IIP}
	\label{gambar:lobi-iip}
\end{figure}
Berdasarkan gambar tersebut, terlihat bahwa belum ada infrastruktur yang dapat memungkinkan sistem untuk melakukan kontrol akses secara otomatis. Saat ini, jika tidak ada penjaga ataupun resepsionis, siapa saja dapat mengakses lift pada lobi yang menuju area lain pada gedung.

Hasil observasi yang dilakukan pada gedung IIP menunjukkan bahwa saat ini, terdapat beberapa sistem keselamatan yang sudah terpasang pada gedung, yaitu:
\begin{enumerate}
    \item Sistem pemantauan, terdiri atas CCTV yang terpasang di berbagai titik pada gedung.
    \item Sistem kebakaran, terdiri atas pendeteksi asap/api, penyiram air, alat pemadam api, dan alarm darurat. 
\end{enumerate}

\section{Perumusan Masalah (\textit{Define})}
Pada tahap ini, informasi yang telah dikumpulkan pada tahap \textit{emphatize} akan dianalisis untuk menemukan dan merumuskan masalah pengguna. Berikut merupakan tahapan \textit{define} yang terdiri atas analisis kebutuhan pengguna, kebutuhan fungsional dan kebutuhan nonfungsional dari sistem yang akan dikembangkan.
\subsection{Identifikasi Masalah Pengguna}
Berdasarkan analisis kondisi saat ini, terdapat dua kelompok pengguna utama dari sistem dengan masalah sebagai berikut:
\begin{enumerate}
    \item   Penghuni gedung, yaitu para karyawan dari perusahaan yang menyewa tempat pada gedung. Kelompok ini membutuhkan sistem yang dapat mengontrol akses masuk ke dalam gedung dengan tingkat keamanan yang tinggi, sehingga orang-orang yang memasuki kawasan mereka hanyalah orang yang telah dikenali secara pasti oleh sistem.
    \item   Pengelola gedung, yaitu tim keamanan dan pengurus gedung yang merupakan karyawan dari Departemen Kawasan Sains dan Teknologi. Kelompok ini membutuhkan sistem kontrol akses yang otomatis dan terintegrasi dengan Sistem Manajemen Bangunan Gedung untuk dapat memenuhi regulasi yang ditetapkan pada Bangunan Gedung Cerdas. Kelompok pengguna ini juga membutuhkan sistem yang tetap sejalan dengan prosedur keselamatan gedung saat terjadi keadaan darurat atau bencana.
\end{enumerate}
Dari kebutuhan yang dimiliki oleh kedua kelompok pengguna ini, dapat dirumuskan sebuah pernyataan kebutuhan atau \textit{problem statement} yaitu "Dibutuhkannya sistem Kontrol akses otomatis dengan keamanan tinggi, dapat terintegrasi dengan Sistem Manajemen Bangunan Gedung, serta tetap sejalan dengan prosedur keselamatan gedung saat terjadi keadaan darurat".
\subsection{Kebutuhan Fungsional}
Berdasarkan perumusan masalah, berikut merupakan analisis kebutuhan fungsional dari sistem yang akan dikembangkan.
\begin{table}[H]
\centering
\begin{tabularx}{\textwidth}{|p{3.5cm}|X|}
  \hline
  \textbf{Nama Kebutuhan}       & \textbf{Penjelasan} \\
  \hline
  Kontrol Akses                 & Sistem harus dapat membatasi dan mengontrol arus akses keluar masuk lift pada lobi gedung. \\
  \hline
  Autentikasi                   & Sistem harus dapat mengenali dan memberikan akses kepada pengguna yang memiliki hak akses memasuki gedung. \\
  \hline
  Manajemen Data                & Pengguna sistem harus dapat mengakses sistem untuk menambahkan, mengubah atau menghapus data mereka yang digunakan dalam sistem.\\
  \hline
  Integrasi Sistem              & Sistem harus dapat melakukan sinkronasi data yang mereka miliki dengan Sistem Manajemen Gedung, dengan tetap memperhatikan regulasi yang berlaku.\\
  \hline
  Keselamatan/\textit{safety}   & Sistem harus memilliki mode darurat, yaitu fitur yang dapat digunakan oleh pengguna untuk menyelamatkan diri mereka saat terjadi keadaan darurat. \\
  \hline
  
\end{tabularx}
\caption{Kebutuhan Fungsional Sistem}
\label{tbl:func-req}
\end{table}
\subsection{Kebutuhan Nonfungsional}
Berdasarkan perumusan masalah, berikut merupakan analisis kebutuhan nonfungsional dari sistem yang akan dikembangkan.
\begin{table}[H]
\centering
\begin{tabularx}{\textwidth}{|p{3.5cm}|X|}
  \hline
  \textbf{Nama Kebutuhan}       & \textbf{Penjelasan} \\
  \hline
  KNF-01: Akurasi                       & Sistem harus dapat mengetahui dan memberikan akses kepada pengguna yang tepat dengan akurasi minimal 90\%   \\
  \hline
  KNF-02: Kapasitas                     & Sistem harus dapat menjalankan fungsi kontrol akses dengan baik untuk kapasitas 1200 pengguna. \\
  \hline
  KNF-03: Waktu Respon                  & Sistem harus dapat memberikan keputusan tentang kontrol akses dalam waktu tiga detik. \\
  \hline
  KNF-04: Keamanan                      & Sistem harus dapat memastikan keamanan data tersimpan sehingga hanya dapat diakses oleh pemilik data ataupun orang yang berwenang. \\
  \hline
  KNF-03: Keandalan                     & Sistem harus dapat beroperasi secara penuh selama 24 jam selama hari kerja (senin s.d jumat). \\
  \hline 
\end{tabularx}
\caption{Kebutuhan Nonfungsional Sistem}
\label{tbl:nonfunc-req}
\end{table}

\section{Analisis Pemilihan Solusi (\textit{Ideate})}
Setelah merumuskan kebutuhan sistem, langkah selanjutnya adalah melakukan analisis dan pencarian terkait berbagai alternatif solusi yang dapat memenuhi kebutuhan sistem. Kemudian setiap alternatif solusi akan dibandingkan dengan analisis kualitatif dan kuantitatif untuk mendapatkan solusi terbaik.
\subsection{Alternatif Solusi}
Berikut merupakan alternatif solusi yang didapatkan untuk setiap kebutuhan fungsional yang ada pada sistem.
\begin{enumerate}
    \item KF-01 Kontrol Akses
    \begin{enumerate}[a.]
        \item \textit{Swing Barrier}, yaitu tipe gerbang yang membuka dan menutup ke arah dalam atau luar.
        \item \textit{Flap Barrier}, yaitu tipe gerbang yang membuka dan menutup dengan menggeser penghalang ke arah samping.
        \item \textit{Tripod Gate}, yaitu tipe gerbang dengan 3 batang besi yang dapat berputar searah saat kunci terbuka.
    \end{enumerate}

    \item KF-02 Pendeteksi Wajah
    \begin{enumerate}[a.]
        \item \textit{Face recognition}, yaitu teknologi autentikasi dengan mendeteksi dan mengenali wajah pengguna.
        \item RFID (\textit{Radio Frequency Identifier}), yaitu teknologi autentikasi yang menggunakan kartu yang memancarkan radio frekuensi tertentu.
        \item Sidik jari, yaitu teknologi autentikasi yang memanfaatkan keunikan pola jari manusia untuk mengenali pengguna.
        \item RFID + Pengenalan Wajah, gabungan dari teknologi pengenalan wajah dan RFID untuk melengkapi kelebihan dan kekurangan masing-masing.
    \end{enumerate}

    \item KF-03 Pengenalan Wajah
    Alternatif Solusi untuk pengenalan wajah adalah kombinasi dari alternatif solusi untuk \textit{Feature Extraction} dan \textit{Matching} 
    %     \item ResNet-50 + ArcFace, 
    %     \item ResNet-50 + MagFace
    %     \item MobileFaceNet + ArcFace
    %     \item MobileFaceNet + MagFace
    % \end{enumerate}

    \textit{Feature Extraction}:
    \begin{enumerate}[a.]
        \item ResNet-50, yaitu arsitektur CNN dengan 50 layer yang menggunakan mekanisme residual learning, menjadi backbone umum untuk pengenalan wajah karena mampu mengekstraksi fitur wajah yang kuat dan stabil.
        \item MobileFaceNet, yaitu ringan dari model pengenalan wajah, cocok digunakan pada perangkat kecil, tetapi masih cukup baik dalam menangkap ciri penting dari wajah.
    \end{enumerate}
    \medskip
    \medskip
    \textit{Loss Function}:
    \begin{enumerate}[a.]
        \item ArcFace, metode pelatihan yang membedakan wajah dengan \"menjauhkan\" jarak antar wajah yang berbeda dan \"mendekatkan\" wajah dari orang yang sama. 
        \item MagFace, metode pelatihan yang tidak hanya membuat model mengenali wajah, tapi juga bisa menilai kualitas foto wajah, sehingga hasil pengenalannya lebih stabil.
    \end{enumerate}
    \item KF-04 Manajemen Data
    \begin{enumerate}[a.]
        \item Aplikasi Web, menggunakan website yang dapat diakses melalui browser untuk pendaftaran.
        \item Aplikasi Desktop, menggunakan aplikasi berbasis desktop untuk perangkat PC resepsionis.
        \item Aplikasi Mobile, menggunakan aplikasi berbasis mobile untuk ponsel pengguna.
    \end{enumerate}

    \item KF-05 Integrasi Sistem
    \begin{enumerate}[a.]
        \item Integrasi berbasis API, memungkinkan sistem saling bertukar data secara langsung melalui HTTP \textit{request} dan \textit{response}.
        \item Integrasi berbasis \textit{Message Queue}, mengirim dan memproses pesan secara asinkron melalui antrian.
        \item Integrasi berbasis \textit{Webhook}, mengirimkan notifikasi otomatis ke sistem lain setiap adanya peristiwa (\textit{event}) tertentu.
    \end{enumerate}

    \item KF-06 Keselamatan/\textit{safety}
    \begin{enumerate}[a.]
        \item Gerbang \textit{fail-safe}, yaitu gerbang yang memiliki kondisi terbuka saat tidak mendapatkan aliran listrik.
        \item Tombol Darurat, yaitu tombol yang dapat membuka gerbang tanpa autentikasi.
        \item Gerbang \textit{fail-safe} + Tombol Darurat, yaitu penggabungan solusi yang memungkinkan gerbang terbuka saat listrik padam atau tombol ditekan.
    \end{enumerate}

\end{enumerate}
\subsection{Analisis Penentuan Solusi}
Untuk menentukan pemilihan solusi terbaik, dilakukan analisis kualitatif dan kuantitatif untuk setiap alternatif solusi. Berikut merupakan analisis kualitatif dari alternatif solusi sistem kontrol akses.
\begin{longtable}{p{0.15\textwidth} p{0.25\textwidth} p{0.55\textwidth}}
\caption{Analisis alternatif solusi}
\label{tab:alternatif_solusi} \\
\toprule
\textbf{Kebutuhan} & \textbf{Opsi Solusi} & \textbf{Analisis Pemilihan} \\
\midrule
\endfirsthead
\caption{Analisis alternatif solusi (lanjutan)} \\
\toprule
\textbf{Kebutuhan} & \textbf{Opsi Solusi} & \textbf{Analisis Pemilihan} \\
\midrule
\endhead
\bottomrule
\endlastfoot

\textbf{KF-1 (Gerbang)} & 1. Tripod Turnstile \newline 2. Flap Barrier \newline 3. \textbf{Swing Barrier} & \textbf{Swing Barrier} dipilih karena memberikan aksesibilitas yang lebih baik (lebar jalur fleksibel untuk barang/kursi roda) dan estetika yang sesuai dengan gedung modern. \\
\midrule
\textbf{KF-2 (Metode)} & 1. Wajah saja \newline 2. RFID saja \newline 3. Sidik Jari \newline 4. \textbf{Wajah + RFID} & \textbf{Wajah + RFID} dipilih sebagai solusi minimal untuk menjamin fleksibilitas (karyawan menggunakan wajah, tamu menggunakan kartu sementara) dan keandalan sistem. \\
\midrule
\textbf{KF-3 (Platform)} & 1. Aplikasi Desktop \newline 2. Aplikasi Mobile \newline 3. \textbf{Aplikasi Web} & \textbf{Aplikasi Web} dipilih karena kemudahan akses (tidak perlu instalasi di sisi pengguna) dan sentralisasi manajemen data yang lebih efisien. \\
\midrule
\textbf{KF-4 (Integrasi)} & 1. \textit{Message Queue} \newline 2. \textbf{Integrasi API} \newline 3. Webhook& \textbf{Integrasi Alarm Api} dipilih agar sistem secara otomatis merespons kondisi kebakaran tanpa menunggu intervensi manusia ("nembak api"). \\
\midrule
\textbf{KF-5 (Safety)} & 1. \textit{Fail-Safe} (Otomatis) \newline 2. Darurat (Manual) \newline 3. \textbf{Kombinasi} & \textbf{Kombinasi (Fail-Safe + Darurat)} dipilih. Mekanisme \textit{fail-safe} membuka kunci saat listrik putus, didukung tombol darurat manual untuk redundansi keamanan. \\
\end{longtable}
Setelah itu, kami melakukan analisis kuantitatif dengan metode \textit{Weighted Scoring Model} (WSM) untuk membandingkan setiap alternatif solusi. berikut merupakan analisis WSM dari setiap alternatif solusi setiap kebutuhan.
\subsubsection{KF-3: Pengenalan Wajah}
Berikut merupakan kriteria penilaian dari fungsionalitas pengenalan wajah.
\begin{table}[H]
\centering
\begin{tabularx}{\textwidth}{|p{3.5cm}|p{2cm}|X|}
  \hline
  \textbf{Kriteria} & \textbf{Bobot (\%)} & \textbf{Alasan} \\
  \hline
  Akurasi & 40\% & ISO/IEC 19795-1:2021 menyatakan bahwa akurasi merupakan aspek utama untuk penilaian sistem biometrik, serta memastikan keamanan pada sistem akses kontrol. \\ 
  \hline
  Kecepatan & 20\% & \textcite{li2024handbook_face_recognition} dalam bukunya membahas bahwa kecepatan sistem melakukan pengenalan penting untuk menghindari antrean.\\ 
  \hline
  Efisiensi perangkat keras & 15\% & Efisiensi perangkat keras dapat menekan biaya yang diperlukan untuk pengembangan sistem. \\ 
  \hline
  Skalabilitas dan Pemeliharaan (\textit{maintainability}) & 15\% & \textcite{pressman2014software} membahas tentang kualitas perangkat lunak dapat diukur dengan berbagai kriteria, diantaranya adalah skalabilitas dan pemeliharaan. Skalabilitas dan pemeliharaan yang baik dapat menekan biaya operasional sistem.  \\ 
  \hline
  Kompleksitas implementasi & 10\% & Kompleksitas yang rendah mempercepat pengembangan sistem dan menghindari risiko bug.\\ 
  \hline
\end{tabularx}
\caption{Bobot penilaian untuk solusi pengenalan wajah}
\label{tab:bobot-fr3}
\end{table}

Berdasarkan kriteria tersebut, berikut merupakan\textit{Weighted Scoring Model} dari setiap alternatif solusi berikut : 
\begin{enumerate}
    \item Solusi 1: ResNet-50 + ArcFace
    \item Solusi 2: ResNet-50 + MagFace
    \item Solusi 3: MobileFaceNet + ArcFace
    \item Solusi 4: MobileFaceNet + MagFace  
\end{enumerate}

\begin{table}[H]
\centering
\caption{Analisis Penentuan Solusi Model Pengenalan Wajah Menggunakan Weighted Scoring Model}
\label{tab:wsm-face-recognition}
\small
\setlength{\tabcolsep}{4pt}

\begin{tabular}{l c c c c c}
\toprule
\textbf{Kriteria Penilaian} 
& \textbf{Bobot} 
& \textbf{Solusi 1} 
& \textbf{Solusi 2} 
& \textbf{Solusi 3} 
& \textbf{Solusi 4} \\
\midrule

Akurasi (40\%) 
& 0.40 
& 5 (2.00) 
& 5 (2.00) 
& 4 (1.60) 
& 4 (1.60) \\

Kecepatan (20\%) 
& 0.20 
& 3 (0.60) 
& 3 (0.60) 
& 5 (1.00) 
& 5 (1.00) \\

Efisiensi Perangkat Keras (15\%) 
& 0.15 
& 3 (0.45) 
& 3 (0.45) 
& 5 (0.75) 
& 5 (0.75) \\

Skalabilitas dan Maintainability (15\%) 
& 0.15 
& 4 (0.60) 
& 4 (0.60) 
& 4 (0.60) 
& 4 (0.60) \\

Kompleksitas Implementasi (10\%) 
& 0.10 
& 3 (0.30) 
& 4 (0.40) 
& 2 (0.20) 
& 3 (0.30) \\

\midrule
\textbf{Total Skor} 
& \textbf{100\%} 
& \textbf{3.95} 
& \textbf{4.05} 
& \textbf{4.15} 
& \textbf{4.25} \\
\bottomrule
\end{tabular}
\end{table}
 
Berdasarkan analisis yang telah dilakukan, solusi terbaik yang dapat kami usulkan adalah:
\begin{enumerate}
    \item Kontrol Akses: \textit{Swing Barrier}
    \item Autentikasi: Kombinasi Pengenalan Wajah + RFID
    \item Manajemen Data: Aplikasi Web
    \item Integrasi Sistem: Integrasi API
    \item Keselamatan/\textit{safety}: Kombinasi \textit{Fail-safe gate} + Tombol Darurat.
\end{enumerate}


