% ==========================================
% BAB I PENDAHULUAN
% ==========================================
\chapter{PENDAHULUAN}
\label{chap:pendahuluan}
% --- Latar Belakang ---
\section{Latar Belakang}
Dalam pengolaan gedung cerdas, aspek keamanan dan keselamatan dalam gedung yang terintegrasi merupakan salah satu aspek yang sangat penting untuk segera diimplementasikan dengan baik. Aktivitas perusahaan, karyawan maupun pengunjung yang beragam dan kompleks membutuhkan adanya suatu sistem kontrol akses yang bukan hanya mengatur keluar masuknya karyawan dan pengunjung, tetapi juga dapat mendukung dan membantu saat terjadinya keadaan darurat atau bencana. Sistem yang dapat membantu mengontrol dan mengenali karyawan atau pengunjung secara tepat dan cepat menjadi salah satu elemen penting dalam mewujudkan hal tersebut. 

Kebutuhan sistem ini juga disebutkan dalam Peraturan Menteri Pekerjaan Umum dan Perumahan Rakyat (PUPR) Nomor 10 Tahun 2023 tentang Bangunan Gedung Cerdas. Regulasi ini menyebutkan pada Pasal 4 ayat (2) bahwa prinsip Bangunan Gedung Cerdas (BGC) harus memiliki sistem-sistem yang bekerja otomatis dan terintegrasi satu sama lain. Peraturan ini juga menyebutkan pada Pasal 5 bahwa elemen BGC harus terintegrasi dalam Sistem Manajemen Bangunan Gedung (\textit{building management system}), dimana elemen terebut diantaranya adalah sistem kontrol akses gedung. 

Science Techno Park (STP) Gedebage, atau biasa disebut juga ITB Innovation Park (IIP) Bandung Technopolis, merupakan infrastruktur yang dibangun untuk mendorong inovasi dan komersialisasi produk-produk teknologi milik Institut Teknologi Bandung (ITB) \autocite{ITB_2024}. Saat ini, Gedung IIP masih belum menerapkan sistem kontrol akses yang otomatis dan dapat diintegrasikan dengan sistem-sistem lain yang ada pada gedung. Hal ini dapat menyebabkan inefisiensi dalam pengelolaan akses karyawan dan penghuni yang diestimasi dapat menampung hingga 1200 orang. Gedung IIP membutuhkan sistem kontrol akses yang mampu memverifikasi identitas karyawan dan penghuni secara cepat dan tepat tanpa menyebabkan antrian, serta memberikan karyawan kemudahan dalam menggunakannya. Namun, penerapan kontrol akses otomatis tersebut tetap harus dilakukan dengan pertimbangan keselamatan penghuni dan karyawan gedung saat keadaan darurat dan bencana.

Kebutuhan akan sistem yang mempertimbangkan keselamatan dibahas dalam Peraturan Pemerintah (PP) Nomor 16 Tahun 2021 tentang Peraturan Pelaksanaan Undang-Undang Nomor 28 Tahun 2002 tentang Bangunan Gedung. Peraturan ini menyebutkan pada Pasal 28 ayat (1) bahwa setiap Bangunan Gedung harus memenuhi ketentuan aspek keselamatan Bangunan gedung. Berdasarkan peraturan ini, dapat disimpulkan bahwa pengembangan fasilitas gedung harus dilakukan dengan keselamatan sebagai pertimbangan. Aspek keselamatan juga dibahas pada Surat Edaran Menteri Pekerjaan Umum Nomor 22/SE/M/2024 tentang Pedoman Penilaian Kinerja Bangunan Gedung Cerdas Tahap Pemanfaatan Dan Pemeriksaan Kinerja Bangunan Gedung Cerdas Tahap Pembongkaran. Surat edaran ini menyebutkan bahwa sistem kontrol akses membutuhkan panduan operasional terkait titik kumpul yang berisi prosedur evakuasi dan pengumpulan titik aman dalam keaadaan darurat. 

Teknologi gerbang \textit{fail-safe} otomatis dengan pengenalan wajah dan mekanisme darurat hadir sebagai solusi yang relevan untuk menjawab kebutuhan tersebut. Dibandingkan metode lain seperti password ataupun \textit{Radio Frequency Identifier} (RFID), teknologi autentikasi menggunakan biometrik termasuk pengenalan wajah menawarkan metode yang lebih efisien dan lebih aman \autocite{vasquez2021facial}. Penerapan teknologi ini memungkinkan adanya sistem kontrol akses yang efisien tanpa hambatan fisik yang berarti. Penggunaan gerbang yang terbuka dalam kondisi tanpa listrik (\textit{fail-safe}) dan mekanisme darurat sangat relevan sebagai langkah untuk mencegah sistem kontrol akses menjadi hambatan dalam penerapan keselamatan Bangunan Gedung saat keadaan darurat.  

Namun, terlepas dengan segala kelebihan yang dimiliki teknologi pengenalan wajah, regulasi pemerintah dalam UU No. 27 Tahun 2022 tentang Perlindungan Data Pribadi (UU PDP) membuat teknologi gerbang \textit{fail-safe} otomatis dengan pengenalan wajah saja memiliki kelemahan yang krusial, dimana pengguna sistem dapat memilih untuk tidak memberikan data wajah mereka sebagai bagian dari privasi data dan keamanan data pribadi, yang menjadikan orang-orang tersebut tidak dapat melakukan akses ke dalam gedung jika diperlukan. Oleh karena itu, dibutuhkan teknologi tambahan yang mampu memberikan fungsionalitas kontrol akses tanpa membutuhkan data biometrik dari penggunanya.

Berdasarkan analisis tersebut, tugas akhir ini mengusulkan perancangan sistem kontrol akses berbasis gerbang otomatis dengan pengenalan wajah dan RFID untuk kondisi darurat dan bencana pada bangunan cerdas. Sistem ini dirancang sebagai solusi menyeluruh yang meliputi perangkat keras dan perangkat lunak sistem. Sistem akan diwujudkan dalam bentuk prototipe fungsional yang menintegrasikan komponen-komponen tertentu yang dapat berperan sebagai kontrol akses pada gedung dengan tetap mempertimbangkan aspek keselamatan.
% --- Rumusan Masalah ---
\section{Rumusan Masalah}
Saat ini, Gedung IIP belum memiliki sistem kontrol akses yang bekerja secara otomatis dan terintegrasi dalam Sistem Manajemen Bangunan Gedung dan hanya memiliki sistem kontrol akses secara manual. Kondisi ini dapat menimbulkan kerentanan keamanan akibat akses keluar masuk gedung yang cukup bebas. Selain itu, kontrol akses yang belum otomatis tidak dapat memenuhi persyaratan Gedung IIP untuk menjadi Bangunan Gedung Cerdas. Namun, Pemasangan kontrol akses yang dilakukan tanpa pertimbangan keselamatan yang matang akan menimbulkan akibat yang fatal jika terjadi situasi darurat pada gedung.

Masalah yang terjadi dapat dirumuskan sebagai berikut: 
\begin{enumerate}
    \item   Bagaimana merancang sistem kontrol akses yang dapat membatasi dan mengenali orang-orang yang memasuki gedung secara otomatis?
    \item	Bagaimana merancang sistem kontrol akses yang dapat terintegrasi dengan \textit{Building Management System} yang dimiliki gedung?
    \item	Bagaimana merancang sistem kontrol akses yang tetap mempertimbangkan aspek keselamatan pada gedung dalam keaadaan darurat?
\end{enumerate}
Untuk dapat memenuhi persyaratan gedung cerdas serta meningkatkan keamanan dan keselamatan, gedung IIP membutuhkan sistem  kontrol akses otomatis yang mampu mengatur dan mengetahui setiap pengunjung gedung, serta memiliki aspek keselamatan yang mumpuni dalam keadaan darurat. Sistem juga harus dapat terintegrasi dengan \textit{Building Management System} yang dimiliki gedung sebagaimana diatur dalam Peraturan Menteri Pekerjaan Umum dan Perumahan Rakyat (PUPR) Nomor 10 Tahun 2023 tentang Bangunan Gedung Cerdas.     

% --- Tujuan ---
\section{Tujuan}
Tujuan dari pelaksanaan tugas akhir ini adalah:
\begin{enumerate}
    \item Mengembangkan sistem kontrol akses yang mampu membatasi dan mengenali orang-orang yang memasuki gedung secara otomatis.
    \item Mengembangkan sistem kontrol akses yang dapat terintegrasi dengan \textit{Building Management System} yang dimiliki gedung. 
    \item Mengembangkan sistem kontrol akses yang tetap sejalan dengan sistem keselamatan yang ada pada gedung saat terjadi keadaan darurat.
\end{enumerate}
Kriteria keberhasilan tugas akhir ini meliputi: 
\begin{enumerate}
    \item   Sistem mampu membatasi dan mengenali orang-orang yang akan memasuki gedung secara otomatis.
    \item	Sistem mampu terintegrasi dengan \textit{Building Management System} yang dimiliki gedung dengan memberikan informasi tentang orang-orang yang telah memasuki gedung melalui gerbang.
    \item   Sistem dapat memenuhi kebutuhan kontrol akses pada gedung dengan tetap memperhatikan aspek keselamatan gedung saat terjadi keadaan darurat dengan menetralisir hambatan yang mungkin terjadi pada gerbang.
\end{enumerate}   

% --- Batasan Masalah ---
\section{Batasan Masalah}
Untuk memastikan pengembangan sistem terarah dan sejalan dengan kebutuhan, dirumuskan batasan masalah sebagai pedoman dalam pelaksanaan tugas akhir ini. batasan masalah tersebut meliputi:
\begin{enumerate}
    \item Tugas akhir ini dikerjakan secara berkelompok yang terdiri dari tiga orang mahasiswa, yaitu Muhammad Rifa Ansyari dengan NIM 18222004, Axelius Davin dengan NIM 18222016, dan Natanael Steven Simangunsong dengan NIM 18222054. Penulis dalam hal ini berfokus pada sistem kontrol akses untuk kondisi darurat dan bencana.
    \item Pengguna sistem yang dilibatkan adalah pihak pengelola gedung IIP beserta salah satu perusahaan yang menggunakan gedung IIP.
    \item Sistem yang dikembangkan hanya mencakup satu unit gerbang sesuai ketersediaan sumber daya, namun dirancang dan dikembangkan sebagai representasi dari keseluruhan sistem.
    \item Sistem akan dikembangkan menggunakan basis data independen yang tidak terintegrasi langsung dengan data yang dimiliki gedung. 
\end{enumerate}
% --- Metodologi Pengerjaan TA ---
\section{Metodologi}
Metodologi yang digunakan pada pengerjaan tugas akhir ini didasarkan pada metodologi \textit{design thinking}. \textit{Design thinking} adalah proses iteratif yang berorientasi pada manusia (\textit{human-centered approach}) dan kolaborasi antara pengembang sistem dan penggunanya. Metodologi ini menghadirkan solusi inovatif berdasarkan bagaimana pengguna berfikir, bertindak, dan merasakan sesuatu. 
\textit{Design thinking} memiliki lima tahapan utama, yaitu \textit{Empathize}, \textit{Define}, \textit{Ideate}, \textit{Prototype}, dan \textit{Test}.
\begin{enumerate}
\item \textit{Empathize} \newline
Tahap ini bertujuan untuk memahami pengguna, baik permasalahan yang dialami, pemahaman dan pengalaman yang mereka miliki di dalam konteks desain sistem. Dalam melakukan \textit{empathize}, pengembang dapat mengumpulkan data secara aktif dengan melakukan \textit{observation} (\textit{observe}), wawancara, survei, dan percakapan langsung (\textit{engage}), ataupun kombinasi antara \textit{observe} dan \textit{engage}. 
\item \textit{Define} \newline
Tahap ini menggunakan hasil yang didapatkan pada tahap \textit{empathize} untuk menemukan dan merumuskan masalah pengguna. Data yang dikumpulkan dapat diolah menjadi pemahaman atau \textit{insight} dan digunakan untuk membentuk \textit{problem statement} atau \textit{point-of-view}. \textit{Problem statement} berperan menjadi petunjuk dalam membuat sistem yang menyelesaikan masalah pengguna. Rumusan masalah yang baik harus spesifik, terukur, dan mampu memicu inspirasi dan kreativitas dalam pencarian solusi.
\item \textit{Ideate} \newline
Tahap ini berfokus pada eksplorasi ide dan solusi yang dapat digunakan sebagai solusi dalam menyelesaikan masalah yang dirumuskan. Pengembang dapat melakukan beragam cara seperti \textit{brainstorming}, \textit{mindmapping}, \textit{sketching}, hingga bahkan \textit{prototyping} untuk mendapatkan solusi-solusi potensial dalam menyelesaikan masalah pengguna.
\item \textit{Prototype} \newline
Tahap ini merupakan proses mewujudkan ide menjadi bentuk nyata yang dapat digunakan untuk mendapatkan jawaban tentang apakah ide yang telah dipilih dapat menyelesaikan permasalahan yang dialami pengguna. \textit{Prototype} yang dihasilkan dapat berupa model fisik, simulasi, sketsa, ataupun wujud lain yang dapat merepresentasikan ide yang dipilih. Tujuan utama dari tahapan ini bukanlah membuat produk akhir, melainkan sebagai langkah dalam mencoba berbagai macam ide dan kemungkinan untuk mendapatkan data dan pemahaman lebih lanjut mengenai permasalahan dan solusi. 
\item \textit{Test} \newline
Tahapan ini bertujuan untuk mendapatkan umpan balik menggunakan \textit{prototype} yang telah dihasilkan untuk mendapatkan informasi dan data baru dari pengguna sistem yang akan dibuat. Tahap \textit{test} berperan dalam menyempurnakan \textit{prototype} dan solusi, mengenali pengguna lebih lanjut, hingga menyempurnakan \textit{point-of-view} atau \textit{problem statement}.
\end{enumerate}
Metode pencarian dan penapisan literatur yang digunakan mencakup hal-hal sebagai berikut:
\begin{enumerate}
    \item Literatur ilmiah berupa buku ataupun artikel digunakan sebagai definisi ataupun dasar teori yang relevan dengan sistem kontrol akses otomatis.
    \item Regulasi pemerintah yang terkait dengan Bangunan Gedung, kontrol akses, Bangunan Gedung Cerdas serta perlindungan data.
    \item Literatur ilmiah berupa jurnal penelitian selama lima tahun terakhir yang digunakan sebagai pertimbangan solusi atau kesenjangan penilitian.
\end{enumerate}
Dokumentasi data yang digunakan mencakup penangkapan citra gambar, citra suara, beserta catatan hasil pengambilan informasi.