% ==========================================
% BAB II STUDI LITERATUR
% ==========================================
\chapter{STUDI LITERATUR}
\label{chap:studi-literatur}
\section{Regulasi Bangunan Gedung Cerdas}
Perancangan sistem kontrol akses pada gedung cerdas tentunya memiliki regulasi dan standar yang berlaku. Di Indonesia, \textcite{PermenPUPR_10_2023} telah menetapkan standar bangunan cerdas dalam Peraturan Menteri Pekerjaan Umum dan Perumahan Rakyat Nomor 10 Tahun 2023 tentang Bangunan Gedung Cerdas, dimana pada Pasal 5 ayat (3) disebutkan bahwa sistem kontrol akses merupakan salah satu elemen bangunan gedung cerdas (BGC) yang harus menggunakan teknologi tinggi dan terintegrasi untuk mewujudkan BGC. Regulasi ini juga digunakan oleh \textcite{mpu22_2024} dalam Surat Edaran Menteri Pekerjaan Umum Nomor 22/SE/M/2024, yang merupakan landasan utama dalam penilaian kinerja bangunan gedung cerdas.  
\section{Regulasi tentang keselamatan Bangunan Gedung}
Regulasi tentang keselamatan Bangunan Gedung dibahas dalam Peraturan Pemerintah Nomor 16 Tahun 2021 tentang Peraturan Pelaksanaan Undang-Undang Nomor 28 Tahun 2002 tentang Bangunan Gedung. Peraturan ini menyebutkan pada Pasal 28 ayat (1) bahwa setiap Bangunan Gedung harus memenuhi ketentuan aspek keselamatan gedung, dimana salah satu nya adalah keadaan darurat berupa kebakaran.

Aspek keselamatan bangunan gedung juga disebutkan pada salah satu kriteria penilaian sistem pada Bangunan Gedung Cerdas. Surat Edaran Menteri Pekerjaan Umum Nomor 22/SE/M/2024 menyebutkan bahwa salah satu Kinerja Unjuk Kerja (KUK) untuk kontrol akses adalah panduan operasional terkait titik kumpul yang penting berisi prosedur evakuasi dan pengumpulan titik aman dalam keadaan darurat, yang merupakan salah satu aspek keselamatan dalam gedung.
\section{Sistem Kontrol Akses}
Sistem kontrol akses adalah metode otomatis yang mengizinkan pihak yang diasumsikan teman untuk memasuki area yang dikontrol, dibatasi atau diamankan dengan penyaringan di portal akses kontrol yang disediakan. Sistem kontrol akses dirancang untuk memastikan bahwa hanya orang yang berwenang atau memenuhi syarat saja yang diizinkan untuk memasuki area ekslusif tersebut \autocite{norman2011electronic}. Lebih lanjut, \textcite{norman2011electronic} menyebut tentang Sistem kontrol akses elektronik yang memanfaatkan komputer, kredensial, pembaca kredensial, dan pintu kunci untuk mengontrol akses secara elektronik. Elemen pada pintu juga termasuk pada alarm dan sensor keluar yang digunakan untuk keadaan tertentu.

\section{Sistem Pengenalan Wajah}
Pengenalan wajah adalah teknologi dalam visi komputer yang digunakan untuk mengidentifikasi seseorang atau suatu objek dari gambar atau video. Pengenalan wajah adalah masalah pengenalan pola visual, dimana wajah sebagai objek tiga dimensi sebagai subjek yang memiliki pencahayaan, pose, dan ekspresi yang bervariasi untuk diidentifikasi berdasarkan gambar dua dimensi yang diambil \autocite{li2024handbook_face_recognition}. Berdasarkan \textcite{li2024handbook_face_recognition} dalam bukunya \citetitle{li2024handbook_face_recognition}, sistem pengenalan wajah terdiri atas empat modul utama, yaitu \textit{detection}, \textit{aligment}, \textit{feature extraction}, dan \textit{matching}, dimana lokasisasi dan normalisasi (tahapan \textit{face detection} dan \textit{aligment}) adalah langkah proses yang dilakukan sebelum pengenalan wajah (\textit{facial feature extraction} dan \textit{matching}).

Pengenalan wajah sudah menjadi biometrik untuk melakukan autentikasi yang digunakan secara luas di berbagai bidang, seperti militer, keuangan, keamanan publik, hingga kehidupan sehari hari. \textcite{Jadhav_2024} menyebutkan bahwa berdasarkan survey dari HID Global pada tahun 2024, jumlah bisnis yang menggunakan biometrik sebagai kontrol akses mereka naik dari 30 persen menjadi 39 persen pada dua tahun terakhir, yang menunjukkan bahwa penggunaan biometrik, termasuk pengenalan wajah sudah mulai diadopsi secara cepat oleh pelaku bisnis.

Teknologi pengenalan wajah telah mengalami perkembangan yang progresif dimana saat ini telah berkembang pendekatan \textit{Deep Learning} yang menawarkan akurasi pengenalan yang tinggi. metode ini telah dibahas oleh \textcite{Gupta} yang menunjukkan bahwa penggunaan pustaka modern seperti OpenCV dengan model \textit{Deep Learning} mampu melakukan deteksi dan pengenalan wajah secara \textit{real-time} dengan tingkat kesalahan rendah, bahkan dalam kondisi lingkungan yang dinamis. 

\section{Gerbang pada Bangunan Gedung sebagai Kontrol Akses}
Terdapat beberapa ketentuan dan rekomendasi tentang bagaimana spesifikasi dari gerbang yang digunakan sebagai kontrol akses dalam bangunan gedung. \textcite{pup_r14_2017} menyebutkan bahwa arah bukaan pintu terpasang pada ruang yang digunakan oleh pengguna dalam jumlah besar harus terbuka searah dengan arah ke luar Bangunan Gedung/ruang. Kemudian dalam lampiran teknisnya, peraturan ini juga menyebutkan bahwa pintu akses (turnstile) memiliki lebar efektif bukaan paling sedikit 60 cm dan untuk disabilitas pintu harus memilik lebar efektif paling sedikit 80 cm. Peraturan ini dapat kita adopsi sebagai referensi menentukan dimensi gerbang yang digunakan pada sistem kontrol akses. Gambar \ref{gambar:contoh-pintu-akses} menunjukkan contoh penerapan pada pintu akses pada lampiran Peraturan Menteri Pekerjaan Umum dan Perumahan Rakyat Nomor 14/PRT/M/2017 Tahun 2017 tentang Persyaratan Kemudahan Bangunan Gedung.
\begin{figure}[H] % pilihan opsi yang disarankan: t = top, b = bottom, h = here
	\centering
  \captionsetup{justification=centering}
    	\includegraphics[width=0.7\textwidth]{image/pintu-akses-2.png}
	\caption{Contoh penerapan desain pada pintu akses (\textit{turnstile})}
	\label{gambar:contoh-pintu-akses}
\end{figure}

Sebagai rekomendasi, \textcite{ding2021winggate} menyebutkan bahwa gerbang bertipe wing lebih baik digunakan dibandingkan dengan gerbang bertipe flat. Pada jurnal tersebut, hasil eksperimen menunjukkan bahwa gerbang bertipe \textit{wing} memiliki kontrol evakuasi dan keramaian yang lebih baik dibandingkan dengan gerbang bertipe \textit{flat}.

\section{Keamanan Data dan Privasi}
Sistem pengenalan wajah atau biometrik memiliki beberapa landasan hukum di Indonesia, terutama terkait dengan beberapa aturan tentang keamanan dan privasi data yang akan dijelaskan sebagai berikut.

\input table/tabelbab2-1.tex

Berdasarkan studi terbaru di Indonesia, beberapa isu hukum dan etika yang muncul bila menerapkan pengenalan wajah diantaranya adalah Persetujuan dan Kesadaran Subjek Data, Privasi dan Pengamanan Data, Kesalahan Identifikasi dan Risiko Penyalahgunaan, hingga Transparansi dan Akuntabilitas.

\section{Pengembangan Sistem Gerbang Otomatis pada KAI}
Layanan \textit{Face Recognition Boarding} KAI merupakan inovasi digital yang diluncurkan oleh PT Kereta Api Indonesia (KAI) untuk mempermudah proses keberangkatan penumpang kereta api jarak jauh. Melalui sistem ini, penumpang tidak lagi perlu menunjukkan kartu identitas (KTP) atau mencetak \textit{boarding pass}, karena proses verifikasi identitas dilakukan secara otomatis menggunakan teknologi pengenalan wajah. Teknologi ini mulai diterapkan di beberapa stasiun besar seperti Stasiun Gambir sejak 1 September 2023, dan akan diperluas secara bertahap ke stasiun-stasiun lainnya di Indonesia.

Untuk dapat menggunakan layanan ini, penumpang diwajibkan melakukan pendaftaran awal melalui aplikasi Access by KAI (sebelumnya dikenal sebagai KAI Access). Proses pendaftaran meliputi unggah data diri berupa Nomor Induk Kependudukan (NIK) sesuai KTP dan foto wajah atau swafoto yang jelas. Data tersebut kemudian akan diverifikasi dan disinkronkan dengan sistem tiket elektronik KAI. Setelah verifikasi berhasil, data wajah penumpang akan tersimpan dalam sistem, dan penumpang dapat langsung melakukan \textit{boarding} hanya dengan menatap kamera di gate yang sudah dilengkapi sensor pengenal wajah.

Pada saat keberangkatan, sistem akan mencocokkan wajah penumpang dengan data tiket dan identitas yang telah terdaftar. Jika data valid dan sesuai, gerbang otomatis terbuka, memungkinkan penumpang masuk tanpa perlu melakukan antri atau pemeriksaan manual. Teknologi ini bertujuan untuk meningkatkan efisiensi dan keamanan proses \textit{boarding}, mengurangi risiko pemalsuan identitas atau penyalahgunaan tiket, serta memberikan pengalaman perjalanan yang lebih modern dan cepat. Namun, KAI sendiri tidak menjadikan layanan \textit{Face Recognition Boarding} KAI sebagai alur utama dalam memasuki kereta. Untuk menjadikan Sistem Gerbang dengan pengenalan wajah sebagai alur utama, sistem tetap harus memberikan alternatif alur akses sebagai langkah penegakan UU No. 27 Tahun 2022 tentang Perlindungan Data Pribadi (UU PDP).
% \section{Kesenjangan}
% Penelitian selama 5 tahun terakhir menunjukkan bahwa belum adanya suatu mekanisme ataupun ketentuan implementasi yang harus dilakukan untuk mengembangkan sistem kontrol akses yang mempertimbangkan keselamatan secara eksplisit. Hal ini membuat diperlukannya penelitian lebih lanjut secara mandiri tentang pengembangan sistem gerbang otomatis berbasis pengenalan wajah untuk kondisi darurat dan bencana. 