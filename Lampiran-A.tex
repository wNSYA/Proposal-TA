\chapter*{LAMPIRAN A. TRANSKRIP WAWANCARA I}
\addcontentsline{toc}{chapter}{LAMPIRAN A - TRANSKRIP WAWANCARA I}

\begin{longtable}{|p{0.05\textwidth}|p{0.35\textwidth}|p{0.55\textwidth}|}
\caption{Transkrip Wawancara dengan Pengelola Gedung (10 November 2025)} \label{tab:wawancara1} \\
\hline
\textbf{No} & \textbf{Pertanyaan / Topik} & \textbf{Jawaban Narasumber} \\ \hline
\endfirsthead
\caption{Transkrip Wawancara dengan Pengelola Gedung (lanjutan)} \\
\hline
\textbf{No} & \textbf{Pertanyaan / Topik} & \textbf{Jawaban Narasumber} \\ \hline
\endhead
\hline
\endlastfoot

1 & Bagaimana sistem kontrol akses saat ini? & Kontrol masih standar menggunakan CCTV. Rencananya dalam bulan ini akan dilakukan pendekatan untuk pemasangan \textit{barrier} atau \textit{access door}. \\ \hline
2 & Bagaimana alur masuk pengunjung dan karyawan saat ini? & Saat ini masih manual. Pengunjung melapor di lobi sebelum menuju lantai tujuan. Rencananya akses kontrol akan dipasang di lobi, area B1, akses lift, dan setiap pintu ruangan. \\ \hline
3 & Apa saja yang dicek oleh petugas keamanan? & Petugas mencatat tujuan lantai dan orang yang dituju secara manual. Ke depannya direncanakan menggunakan \textit{face recognition}, sidik jari, dan RFID. \\ \hline
4 & Apakah ada rencana spesifik untuk jenis gerbang? & Ya, direncanakan menggunakan \textit{swing barrier}. Akses masuk-keluar akan menggunakan pengenalan wajah (prioritas untuk karyawan) dan kartu akses/RFID (untuk tamu). \\ \hline
5 & Bagaimana spesifikasi kinerja gerbang yang diharapkan? & Akurasi pembacaan wajah diharapkan maksimal di angka 90\%. Kecepatan input (\textit{speed}) juga harus diatur agar tidak menghambat alur. \\ \hline
6 & Berapa jumlah gerbang yang akan dipasang? & Dengan lebar area 3,4 meter, direncanakan dipasang 3 unit \textit{barrier} (3 jalur masuk dan 3 jalur keluar) untuk menutup celah agar orang tidak bisa menyelinap. \\ \hline
7 & Apakah sistem absensi akan diintegrasikan? & Absensi diserahkan ke masing-masing \textit{tenant}/perusahaan di setiap lantai. Pengelola gedung hanya fokus pada akses masuk utama. \\ \hline
8 & Berapa estimasi kapasitas pengguna gedung? & Kapasitas maksimal gedung diperkirakan mencapai 1.200 orang. \\ \hline
9 & Bagaimana prosedur penerimaan tamu saat ini (SOP)? & Tamu melapor ke sekuriti di B1 atau resepsionis di lobi. Petugas akan menghubungi PIC di lantai tujuan untuk konfirmasi izin masuk. \\ \hline
10 & Bagaimana rencana SOP setelah sistem otomatis terpasang? & Tamu menukarkan KTP dengan kartu akses (RFID) di resepsionis untuk membuka gerbang dan akses lift. Karyawan wajib menggunakan wajah (\textit{face recognition}). Sekuriti akan menggunakan sidik jari untuk akses mereka. \\ \hline
11 & Bagaimana mekanisme keselamatan saat bencana (kebakaran/gempa)? & Gedung sudah memiliki \textit{Master Control Fire Alarm} (MCFA). Jika terjadi bencana, sistem gerbang (\textit{barrier}) harus tersetting untuk terbuka otomatis (\textit{fail-safe}) agar tidak menghalangi evakuasi. \\ \hline
\end{longtable}
% \section{Perangkat Lunak untuk Akuisisi Data dari Sensor Ultrasonik}
% \lstinputlisting[language=Python, caption=source code untuk akuisisi data dari sensor ultrasonik]{code/ultrasonic_data_acquisition.py}  
% \section{Perangkat Lunak untuk Akuisisi Data dari Sensor Suhu dan Kelembaban}
% \lstinputlisting[language=Python, caption=source code untuk akuisisi data dari sensor suhu dan kelembaban]{code/temp_humidity_data_acquisition.py}  
