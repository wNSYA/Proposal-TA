\begin{longtable}{l p{0.7\textwidth}}
\caption{Rencana implementasi prototipe}
\label{tab:rencana-implementasi-detail} \\
\toprule
\textbf{Komponen atau Aspek} & \textbf{Deskripsi dan Spesifikasi} \\
\midrule
\endfirsthead
\caption{Rencana implementasi prototipe (lanjutan)} \\
\toprule
\textbf{Komponen atau Aspek} & \textbf{Deskripsi dan Spesifikasi} \\
\midrule
\endhead
\bottomrule
\endlastfoot

\textit{Perangkat Keras (x2 Set)} & \\
Unit Pemrosesan & 2 unit Raspberry Pi 4 Model B (4GB RAM). Dipilih karena keseimbangan antara performa untuk \textit{edge computing} dan ketersediaan \textit{port} GPIO. \\
Sensor Kamera & 2 unit Raspberry Pi Camera Module v3. Digunakan untuk akuisisi citra wajah dengan fitur HDR. \\
Layar Display & 2 unit LCD 5 inch HDMI. Untuk antarmuka visual pengguna. \\
Aktuator Kunci & Solenoid Door Lock 12V yang dikontrol melalui 5V Relay Module (Simulasi). \\
Pendukung & Power Supply 5V 3A, Fan Cooling, SSD Eksternal 128GB, dan kabel \textit{jumper}. \\
\midrule
\textit{Perangkat Lunak} & \\
Sistem Operasi & Raspberry Pi OS berbasis Debian. \\
Bahasa & Python 3. \\
Pustaka Utama & OpenCV, Dlib, RPi.GPIO. \\
\midrule
\textit{Lingkungan} & \\
Lokasi & Prototipe akan diuji pada simulasi pintu masuk di Laboratorium X, Gedung IIP. \\
Konfigurasi & Perangkat akan dipasang pada ketinggian rata-rata wajah orang berdiri, yaitu sekitar 160 cm sampai 170 cm. \\
\midrule
\textit{Estimasi Biaya} & \\
Perangkat Keras Utama & Rp 5.000.000 (untuk 2 unit sistem wajah) \\
Komponen Pendukung & Rp 500.000 \\
\midrule
\textit{Total Estimasi} & \textbf{Rp 5.500.000} \\
\end{longtable}