\begin{table}[ht]
\centering
\caption{Analisis risiko dan mitigasi proyek}
\label{tab:analisis-risiko-detail}
\begin{tabularx}{\textwidth}{ l >{\raggedright\arraybackslash}X >{\raggedright\arraybackslash}X >{\raggedright\arraybackslash}X }
\toprule
\textbf{No.} & \textbf{Risiko} & \textbf{Dampak} & \textbf{Tindakan Mitigasi} \\
\midrule

1. &
Risiko Teknis: \newline
Akurasi pengenalan wajah rendah di bawah 90\%. &
Gagal memenuhi KNF-2. Pengguna terdaftar ditolak. &
1. Melakukan \textit{data augmentation}. \newline
2. Menggunakan lampu tambahan (\textit{fill light}). \\
\midrule

2. &
Risiko Teknis: \newline
Waktu respon lambat lebih dari 3 detik. &
Gagal memenuhi KNF-1. Menyebabkan antrian. &
1. Optimasi kode. \newline
2. Menggunakan algoritma pencarian vektor cepat (Annoy). \\
\midrule

3. &
Risiko Operasional: \newline
Kegagalan fungsi akibat \textit{backlight}. &
Akurasi menurun drastis pada jam tertentu. &
1. Menggunakan kamera HDR. \newline
2. Mengatur posisi kamera. \\
\midrule

4. &
Risiko Keamanan: \newline
\textit{Spoofing attack} menggunakan foto HP. &
Akses ilegal. &
1. Implementasi deteksi kedipan mata. \newline
2. Pengawasan oleh sekuriti (mitigasi prosedural). \\

5. &
Risiko Pengembangan: \newline
Jadwal aktual tidak sesuai dengan estimasi yang ditetapkan &
 Keterlambatan Waktu Penyelesaian &
1. Prioritisasi Fitur  \newline
2. Penjadwalan Ulang yang lebih realistis.\\

\bottomrule
\end{tabularx}
\end{table}
