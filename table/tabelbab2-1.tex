\begin{table}[H]
\centering
\begin{tabular}{ | p{4cm} | p{9cm} | }
  \hline
  \textbf{Nama Aturan} & \textbf{Keterangan} \\
  \hline
  UU No. 27 Tahun 2022 tentang Perlindungan Data Pribadi (UU PDP) & Mengatur data pribadi, termasuk data spesifik/biometrik; mengharuskan persetujuan, hak subjek data, keamanan penyimpanan, hak koreksi/hapus data, dan sanksi jika disalahgunakan. \\
  \hline
  UU No. 19 Tahun 2016 tentang Informasi dan Transaksi Elektronik (UU ITE) & Mengatur informasi elektronik, penyebaran data, penyalahgunaan data, distribusi konten, dan hukum atas tindakan‐tindakan berbasis elektronik; termasuk perlindungan data pribadi. \\
  \hline
  Peraturan Menteri Komunikasi dan Digital nomor 7 Tahun 2025 & Mengenai Pemanfaatan Teknologi Modul Identitas Pelanggan Melekat (eSIM) dalam Telekomunikasi. Termasuk penggunaan data biometrik (wajah atau sidik jari) untuk registrasi. Berlaku penuh mulai 2027.\\
  \hline
\end{tabular}
\caption{Peraturan terkait pengenalan wajah}
\label{tbl:peraturan-pengenalan-wajah}
\end{table}