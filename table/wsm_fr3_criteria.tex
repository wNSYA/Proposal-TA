\begin{table}[H]
\centering
\begin{tabularx}{\textwidth}{|p{3.5cm}|p{2cm}|X|}
  \hline
  \textbf{Kriteria} & \textbf{Bobot (\%)} & \textbf{Alasan} \\
  \hline
  Akurasi & 40\% & ISO/IEC 19795-1:2021 menyatakan bahwa akurasi merupakan aspek utama untuk penilaian sistem biometrik, serta memastikan keamanan pada sistem akses kontrol. \\ 
  \hline
  Kecepatan & 20\% & \textcite{li2024handbook_face_recognition} dalam bukunya membahas bahwa kecepatan sistem melakukan pengenalan penting untuk menghindari antrean.\\ 
  \hline
  Efisiensi perangkat keras & 15\% & Efisiensi perangkat keras dapat menekan biaya yang diperlukan untuk pengembangan sistem. \\ 
  \hline
  Skalabilitas dan Pemeliharaan (\textit{maintainability}) & 15\% & \textcite{pressman2014software} membahas tentang kualitas perangkat lunak dapat diukur dengan berbagai kriteria, diantaranya adalah skalabilitas dan pemeliharaan. Skalabilitas dan pemeliharaan yang baik dapat menekan biaya operasional sistem.  \\ 
  \hline
  Kompleksitas implementasi & 10\% & Kompleksitas yang rendah mempercepat pengembangan sistem dan menghindari risiko bug.\\ 
  \hline
\end{tabularx}
\caption{Bobot penilaian untuk solusi pengenalan wajah}
\label{tab:bobot-fr3}
\end{table}
