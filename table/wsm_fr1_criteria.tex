\begin{table}[H]
\centering
\caption{Kriteria penilaian kontrol akses fisik}
\label{tab:kriteria-kontrol-akses}
\begin{tabularx}{\textwidth}{|p{3.5cm}|p{2cm}|X|}
  \hline
  \textbf{Kriteria} & \textbf{Bobot (\%)} & \textbf{Alasan} \\
  \hline
  Keamanan Fisik & 35\% & Berdasarkan ISO/IEC 21964:2021, pengamanan fisik harus mencegah akses tidak sah dan meminimalkan risiko pelanggaran keamanan, sehingga menjadi aspek terpenting dalam evaluasi sistem kontrol akses. \\ 
  \hline
  Kecepatan Melalui Gerbang (Throughput) & 25\% & Tingkat throughput berpengaruh langsung pada kelancaran arus pengguna. Sistem kontrol akses perlu mempertahankan flow rate yang tinggi untuk menghindari penumpukan dan meningkatkan efisiensi operasional bangunan.\\ 
  \hline
  Kenyamanan dan Aksesibilitas & 15\% & Peraturan Menteri PUPR No. 14/PRT/M/2017 menekankan pentingnya kemudahan penggunaan, termasuk aksesibilitas bagi penyandang disabilitas, sehingga gerbang harus aman namun tetap nyaman. \\ 
  \hline
  Efisiensi Energi dan Perawatan & 15\% & perangkat fisik yang hemat energi dan mudah dipelihara dapat mengurangi total biaya kepemilikan (total cost of ownership) pada sistem gedung.  \\ 
  \hline
  Keandalan Operasional & 10\% & Perangkat kontrol akses harus dapat beroperasi stabil dalam kondisi normal maupun darurat, serta mendukung keamanan evakuasi.\\ 
  \hline
\end{tabularx}
\end{table}