\begin{table}[H]
\centering
\caption{Desain Pengujian dan Evaluasi Sistem}
\label{tab:desain-pengujian-detail}
\begin{tabularx}{\textwidth}{ l >{\raggedright\arraybackslash}X >{\raggedright\arraybackslash}X }
\toprule
\textbf{Kriteria} & \textbf{Metode Verifikasi atau Validasi} & \textbf{Parameter Keberhasilan} \\
\midrule
\textit{Verifikasi Fungsional} & & \\
KF-1 s.d. KF-4 & Pengujian Fungsional (\textit{Black-box}) \newline
Skenario: \newline
1. Uji pengguna terdaftar. \newline
2. Uji pengguna tidak terdaftar.
& 1. Skenario 1: Kunci harus terbuka. \newline
2. Skenario 2: Kunci harus tetap tertutup. \\
\midrule
\textit{Validasi Nonfungsional} & & \\
KNF-3 (Waktu Respon) & Pengujian Waktu Respon \newline
Mengukur waktu dari saat wajah terdeteksi penuh hingga sinyal dikirim ke aktuator.
& Waktu rata-rata dari 20 kali percobaan harus kurang dari 3 detik. \\
\midrule
KNF-1 (Akurasi) & Pengujian Akurasi \newline
Membuat \textit{dataset} uji 10 pengguna terdaftar dan 5 pengguna tidak terdaftar.
& \textit{False Acceptance Rate} (FAR) < 0,1\% dan \textit{False Rejection Rate} (FRR) < 1\%. \\
\midrule
KNF-5 (Keandalan) & Pengujian Konsistensi \newline
Uji berulang sebanyak 20 kali.
& Sistem harus konsisten 100\% dalam memberikan keputusan akses. \\
\bottomrule
\end{tabularx}
\end{table}
