\begin{table}[H]
\centering
\caption{Desain Pengujian dan Evaluasi Sistem}
\label{tab:desain-pengujian-detail}
\begin{tabularx}{\textwidth}{ l >{\raggedright\arraybackslash}X >{\raggedright\arraybackslash}X }
\toprule
\textbf{Kriteria} & \textbf{Metode Verifikasi atau Validasi} & \textbf{Parameter Keberhasilan} \\
\midrule
\textit{Verifikasi Fungsional} & & \\
KF-5 (API) & Pengujian \textit{endpoint} API menggunakan Postman. & API mengembalikan respon kode 200 OK dan format JSON yang valid. \\
KF-6 (Safety) & Simulasi pemutusan daya listrik dan penekanan tombol darurat. & Kunci gerbang terlepas (mode bebas dorong) secara instan. \\
\midrule
\textit{Validasi Nonfungsional} & & \\
KNF-5 (Keandalan) & Uji operasional (\textit{stress test}) selama sehari. & Tidak terjadi \textit{crash}, \textit{overheat}, atau kegagalan fungsi selama pengujian. \\
\bottomrule
\end{tabularx}
\end{table}
