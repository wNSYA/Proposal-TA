\begin{table}[H]
\centering
\caption{Kriteria penilaian solusi pendeteksi wajah}
\label{tab:kriteria-pendeteksi-wajah}
\begin{tabularx}{\textwidth}{|p{3.5cm}|p{2cm}|X|}
\hline
\textbf{Kriteria} & \textbf{Bobot(\%)} & \textbf{Alasan} \\
\hline
Akurasi Deteksi & 40\% &
Pada sistem kontrol akses, akurasi deteksi wajah sangat penting karena kesalahan deteksi (\textit{missed detection} atau \textit{false detection}) dapat menghambat proses autentikasi.  \\
\hline
Kecepatan (\textit{Latency} / FPS) & 25\% &
Sistem gerbang membutuhkan respons \textit{real-time} agar tidak terjadi antrean. \\
\hline
Efisiensi Perangkat (Resource Usage) & 15\% &
Model deteksi yang efisien dapat berjalan pada edge device tanpa memerlukan perangkat keras mahal.\\
\hline
Robustness Kondisi Lapangan & 10\% &
Detektor harus tahan terhadap variasi pose, pencahayaan, dan occlusion.  \\
\hline
Kompleksitas Implementasi & 10\% &
Kompleksitas memengaruhi waktu integrasi dan risiko bug. Model dengan arsitektur sederhana lebih mudah di-deploy pada sistem pintu otomatis tanpa modifikasi pipeline besar. \\
\hline
\end{tabularx}
\end{table}