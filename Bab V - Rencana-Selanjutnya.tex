% ==========================================
% BAB V RENCANA SELANJUTNYA
% ==========================================
\chapter{RENCANA SELANJUTNYA}
\label{chap:rencana-selanjutnya}
Bab ini menjelaskan langkah implementasi sistem yang akan dilakukan kedepannya. Rencana ini dijabarkan dalam bentuk linimasa pekerjaan, yang kemudian dilanjutkan dengan rencana implementasi, desain pengujian, serta analisis risiko. 
\section{Rencana Implementasi}
Pengerjaan tugas akhir direncanakan berlangsung selama 14 bulan, dimulai dari September 2025 untuk tahap studi awal dan proposal hingga Desember 2026. Implementasi sistem fisik akan dimulai pada Januari 2026. Linimasa pengerjaan disajikan dalam bentuk bagan Gantt pada Tabel \ref{tab:linimasa}.

\begin{table}[H]
\centering
\caption{Gantt Chart Rencana Pengerjaan Tugas Akhir}
\label{tab:linimasa}
\footnotesize
\setlength{\tabcolsep}{3pt}
\resizebox{\textwidth}{!}{%
\begin{tabular}{l c c c c c c}
\toprule
\textbf{Tahapan Kegiatan} & \textbf{Sep--Okt '25} & \textbf{Nov--Des '25} & \textbf{Jan--Mar '26} & \textbf{Apr--Jun '26} & \textbf{Jul--Sep '26} & \textbf{Okt '26} \\
\midrule

1. Perencanaan dan Persiapan 
  & \rule{2cm}{4pt} & \rule{2cm}{4pt} &  &  &  &  \\

Penyusunan Proposal (Studi Awal) 
  & \rule{2cm}{4pt} &  &  &  &  &  \\

Studi Lanjut dan Persiapan Perangkat 
  &  & \rule{2cm}{4pt} &  &  &  &  \\

\midrule

2. Implementasi dan Pengembangan 
  &  &  & \rule{2cm}{4pt} & \rule{2cm}{4pt} &  &  \\

Pengembangan Perangkat Keras 
  &  &  & \rule{2cm}{4pt} &  &  &  \\

Pengembangan Perangkat Lunak 
  &  &  & \rule{2cm}{4pt} & \rule{2cm}{4pt} &  &  \\

Integrasi Sistem 
  &  &  &  & \rule{2cm}{4pt} &  &  \\

\midrule

3. Pengujian dan Evaluasi 
  &  &  &  &  & \rule{2cm}{4pt} &  \\

Pengujian dan Analisis Hasil 
  &  &  &  &  & \rule{2cm}{4pt} &  \\

\midrule

4. Penulisan dan Finalisasi 
  &  &  &  & \rule{2cm}{4pt} & \rule{2cm}{4pt} & \rule{2cm}{4pt} \\

Penulisan Laporan (Bab 1--5) 
  &  &  &  & \rule{2cm}{4pt} & \rule{2cm}{4pt} &  \\

Penulisan Bab 6 dan Finalisasi 
  &  &  &  &  &  & \rule{2cm}{4pt} \\

\bottomrule
\end{tabular}
}
\end{table}


Lingkup implementasi pada tahap ini difokuskan pada satu gerbang akses dengan kemampuan dua arah, yaitu masuk dan keluar. Oleh karena itu, dibutuhkan total dua set perangkat pengenalan wajah. Tabel \ref{tab:rencana-implementasi-detail} merincikan kebutuhan perangkat keras dan estimasi biaya untuk implementasi tersebut.
\begin{longtable}{l p{0.55\textwidth}} 
\caption{Rencana implementasi prototipe}
\label{tab:rencana-implementasi-detail} \\
\toprule
\textbf{Komponen atau Aspek} & \textbf{Deskripsi dan Spesifikasi} \\
\midrule
\endfirsthead
\caption{Rencana implementasi prototipe (lanjutan)} \\
\toprule
\textbf{Komponen atau Aspek} & \textbf{Deskripsi dan Spesifikasi} \\
\midrule
\endhead
\bottomrule
\endlastfoot

\textit{Perangkat Keras (Sistem Wajah)} & \\
Unit Pemrosesan & 2 unit Raspberry Pi 4 Model B (4GB RAM). Dipilih karena keseimbangan antara performa untuk \textit{edge computing} dan ketersediaan \textit{port} GPIO. \\
Sensor Kamera & 2 unit Raspberry Pi Camera Module v3 (Sony IMX708, 12MP, HDR, Autofocus). \\
Layar Display & 2 unit LCD 5 inch HDMI. Untuk antarmuka visual pengguna. \\
Modul Kontrol & 5V Relay Module (2-Channel) untuk integrasi sinyal buka ke gerbang. \\
Pendukung & Power Supply 5V 3A, Fan Cooling, SSD Eksternal 128GB, dan kabel \textit{jumper}. \\
\midrule

\textit{Gerbang Fisik (Swing Barrier)} & \\
Model & \textbf{IMS347 Slim Swing Turnstile} (Entragate). \\
Dimensi Unit & Panjang 1400 mm $\times$ Lebar 185 mm $\times$ Tinggi 1020 mm. \\
Lebar Jalur & 600 mm - 900 mm (Fleksibel untuk aksesibilitas). \\
Material & Kabinet: Stainless Steel SUS 304. Lengan: Akrilik/Tempered Glass. \\
Mekanisme & DC Brushless Motor dengan kecepatan buka 0.2 detik. \\
Fitur Keamanan & \textit{Anti-tailgating}, sensor inframerah anti-jepit, dan mode darurat (\textit{fail-safe}) otomatis terbuka saat listrik padam. \\
Interface & Input sinyal \textit{Dry Contact} (Relay) dan komunikasi RS485/RS232. \\
\midrule

\textit{Perangkat Lunak} & \\
Sistem Operasi & Raspberry Pi OS (Debian Bookworm 64-bit). \\
Bahasa & Python 3.11. \\
Pustaka Utama & OpenCV, TensorFlow Lite (untuk FaceNet), Ultralytics (untuk YOLOv8), RPi.GPIO. \\
\midrule

\textit{Lingkungan Pengujian} & \\
Lokasi & Simulasi pintu masuk di Lobi, Gedung IIP. \\
Konfigurasi & Kamera dipasang pada ketinggian 160-170 cm pada tiang gerbang. \\
\midrule

\textit{Estimasi Biaya} & \\
Perangkat Elektronik & Rp 5.500.000 (Raspberry Pi, Kamera, Layar, dll). \\
Unit Gerbang Fisik & Rp 18.000.000 (1 Jalur Swing Barrier IMS347). \\
\midrule
\textit{Total Estimasi} & \textbf{Rp 23.500.000} \\
\end{longtable}

\section{Rencana Pengujian dan Evaluasi}
Pengujian dan evaluasi akan dilakukan untuk memverifikasi kebutuhan fungsional dan memvalidasi kebutuhan nonfungsional. Metode pengujian dirangkum pada Tabel \ref{tab:desain-pengujian-detail}.

\begin{table}[H]
\centering
\caption{Desain Pengujian dan Evaluasi Sistem}
\label{tab:desain-pengujian-detail}
\begin{tabularx}{\textwidth}{ l >{\raggedright\arraybackslash}X >{\raggedright\arraybackslash}X }
\toprule
\textbf{Kriteria} & \textbf{Metode Verifikasi atau Validasi} & \textbf{Parameter Keberhasilan} \\
\midrule
\textit{Verifikasi Fungsional} & & \\
KF-5 (API) & Pengujian \textit{endpoint} API menggunakan Postman. & API mengembalikan respon kode 200 OK dan format JSON yang valid. \\
KF-6 (Safety) & Simulasi pemutusan daya listrik dan penekanan tombol darurat. & Kunci gerbang terlepas (mode bebas dorong) secara instan. \\
\midrule
\textit{Validasi Nonfungsional} & & \\
KNF-5 (Keandalan) & Uji operasional (\textit{stress test}) selama sehari. & Tidak terjadi \textit{crash}, \textit{overheat}, atau kegagalan fungsi selama pengujian. \\
\bottomrule
\end{tabularx}
\end{table}


\section{Analisis Risiko dan Mitigasi}
Analisis risiko dilakukan untuk mengidentifikasi potensi masalah selama implementasi dan pengujian, beserta tindakan mitigasi yang disiapkan sebagaimana tercantum pada Tabel \ref{tab:analisis-risiko-detail}.

\begin{table}[ht]
\centering
\caption{Analisis risiko dan mitigasi proyek}
\label{tab:analisis-risiko-detail}
\begin{tabularx}{\textwidth}{ l >{\raggedright\arraybackslash}X >{\raggedright\arraybackslash}X >{\raggedright\arraybackslash}X }
\toprule
\textbf{No.} & \textbf{Risiko} & \textbf{Dampak} & \textbf{Tindakan Mitigasi} \\
\midrule

1. &
Risiko Teknis: \newline
Akurasi pengenalan wajah rendah di bawah 90\%. &
Gagal memenuhi KNF-2. Pengguna terdaftar ditolak. &
1. Melakukan \textit{data augmentation}. \newline
2. Menggunakan lampu tambahan (\textit{fill light}). \\
\midrule

2. &
Risiko Teknis: \newline
Waktu respon lambat lebih dari 3 detik. &
Gagal memenuhi KNF-1. Menyebabkan antrian. &
1. Optimasi kode. \newline
2. Menggunakan algoritma pencarian vektor cepat (Annoy). \\
\midrule

3. &
Risiko Operasional: \newline
Kegagalan fungsi akibat \textit{backlight}. &
Akurasi menurun drastis pada jam tertentu. &
1. Menggunakan kamera HDR. \newline
2. Mengatur posisi kamera. \\
\midrule

4. &
Risiko Keamanan: \newline
\textit{Spoofing attack} menggunakan foto HP. &
Akses ilegal. &
1. Implementasi deteksi kedipan mata. \newline
2. Pengawasan oleh sekuriti (mitigasi prosedural). \\

5. &
Risiko Pengembangan: \newline
Jadwal aktual tidak sesuai dengan estimasi yang ditetapkan &
 Keterlambatan Waktu Penyelesaian &
1. Prioritisasi Fitur  \newline
2. Penjadwalan Ulang yang lebih realistis.\\

\bottomrule
\end{tabularx}
\end{table}
