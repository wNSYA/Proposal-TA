%! TEX program = xelatex
% 
% Template Proposal Tugas Akhir
% Program Studi Sistem dan Teknologi Informasi
% Sekolah Teknik Elektro dan Informatika
% Institut Teknologi Bandung
% 
% Dibuat oleh: IGB Baskara Nugraha 
% Email: baskara@itb.ac.id 
% 
% Last updated: 20 Oktober 2025
%
% Petunjuk penggunaan:
% 1. Ada 2 file utama, yaitu ProposalTA.tex (file ini) dan daftar-pustaka.bib (file daftar pustaka).
% 2. Sunting ProposalTA.tex sesuai dengan kebutuhan Anda.
% 3. Sunting atau generate isi daftar-pustaka.bib dengan referensi yang Anda gunakan, sesuai dengan format BibLaTeX.
% 4. Simpan kedua file tersebut dalam satu folder yang sama.
% 5. Kompilasi file ProposalTA.tex menggunakan XeLaTeX dan Biber (lihat urutan cara kompilasi di bawah).
% 6. Hasil kompilasi adalah file ProposalTA.pdf yang siap dicetak.
% 
% Urutan cara kompilasi (melalui command line):
% 1. xelatex ProposalTA.tex
% 2. biber ProposalTA      
% 3. xelatex  ProposalTA.tex
% 4. xelatex  ProposalTA.tex
%
% Catatan:
% - Pastikan Anda telah menginstal paket-paket LaTeX yang diperlukan, termasuk
%   biblatex-chicago dan fontspec.
% - Gunakan editor LaTeX yang mendukung XeLaTeX, seperti TeXstudio, Overleaf, atau lainnya.
% - Jika meenggunakan Visual Studio Code sebagai editor, pastikan mengatur "latex-workshop.latex.tools" dan
%   "latex-workshop.latex.recipes" untuk mendukung XeLaTeX dan Biber dengan cara menambahkan konfigurasi berikut:
%   "latex-workshop.latex.tools": [ 
%       {
%           "name": "xelatex",
%           "command": "xelatex",
%           "args": [
%               "-synctex=1",
%               "-interaction=nonstopmode",
%               "-file-line-error",
%               "%DOC%"
%           ]
%       },
%       {
%           "name": "biber",
%           "command": "biber",
%           "args": [
%               "%DOCFILE%"
%           ]
%       }
%   ],
%   "latex-workshop.latex.recipes": [
%       {
%           "name": "xelatex -> biber -> xelatex*2",
%           "tools": [
%               "xelatex",
%               "biber",
%               "xelatex",
%               "xelatex"
%           ]
%       }
%   ]
% - Untuk referensi lebih lanjut tentang penggunaan BibLaTeX dengan gaya Chicago, silakan merujuk ke dokumentasi resmi BibLaTeX.
%   https://ctan.org/pkg/biblatex-chicago
% - Untuk referensi lebih lanjut tentang penggunaan XeLaTeX dan fontspec, silakan merujuk ke dokumentasi resmi fontspec.
%   https://ctan.org/pkg/fontspec
% - Selamat menyusun proposal tugas akhir Anda!
%
\documentclass[12pt,a4paper,oneside]{book}

% ==========================================
% BASIC PACKAGES
% ==========================================
\usepackage[utf8]{inputenc} % for UTF-8 encoding
\usepackage{fontspec} % for font selection
\setmainfont{Times New Roman} % set main font to Times New Roman
\usepackage[a4paper, left=4cm, right=3cm, top=3cm, bottom=3cm]{geometry} % set page margins
\usepackage[indonesian]{babel} % untuk bahasa Indonesia
\usepackage{csquotes} % for context-sensitive quotation facilities
\usepackage{setspace} % for line spacing
\onehalfspacing % spasi 1.5
\usepackage{graphicx} % for images
\usepackage{caption} % for customizing captions
\usepackage{subcaption} % for sub-figures
\usepackage{hyperref} % for hyperlinks
\usepackage{titlesec} % for customizing titles
\usepackage{tocloft} % for customizing table of contents
\usepackage{lipsum} % for dummy text (lorem ipsum text)
\usepackage{floatrow} % for customizing float (figure and table) positions
\usepackage{listings} % for code listing
\usepackage{amsmath} % for math
\usepackage{amssymb} % for math symbols
\usepackage[shortlabels]{enumitem} % for customizing lists
\setlist[enumerate]{nosep, topsep=-10pt} % mengurangi spasi antar item dan atas bawah daftar
\setlist[itemize]{nosep, topsep=-10pt} % mengurangi spasi antar item dan atas bawah daftar
\usepackage[skip=12pt]{parskip}
\usepackage{longtable}
\usepackage{booktabs}
\usepackage[bahasai]{datetime2}


\setcounter{tocdepth}{4} % kedalaman daftar isi sampai subsubbab
\setcounter{secnumdepth}{4} % kedalaman penomoran sampai subsubbab


% ==========================================
% SITASI DAN DAFTAR PUSTAKA (MENGGUNAKAN CHICAGO MANUAL OF STYLE)
% ==========================================
\usepackage[
    backend=biber,
    authordate,
    language=english,
    autolang=other
]{biblatex-chicago}

\addbibresource{daftar-pustaka.bib}

% ==========================================
% Ubah istilah bahasa Inggris di daftar pustaka ke Bahasa Indonesia
% ==========================================
\DefineBibliographyStrings{english}{
  and          = {dan},
  andothers    = {dkk.},
  editor       = {penyunting},
  editors      = {penyunting},
  translator   = {penerjemah},
  byeditor     = {disunting oleh},
  bytranslator = {diterjemahkan oleh},
  in           = {dalam},
  edition      = {edisi},
  pages        = {hal.},
  page         = {hal.},
  volume       = {vol.},
  number       = {no.},
  urlseen      = {diakses pada},
  url          = {tautan},
}

% ==========================================
% Pastikan \cite() menampilkan (Penulis Tahun)
% ==========================================
\let\oldcite\cite
\renewcommand{\cite}{\parencite}

% ==========================================
% Atur pemisah nama penulis agar lebih natural dalam Bahasa Indonesia
% ==========================================
\renewcommand*{\finalandcomma}{} % hilangkan koma sebelum 'dan'


% ==========================================
% TAMPILAN
% ==========================================
\hypersetup{
    colorlinks=true,
    linkcolor=black,
    citecolor=black,
    urlcolor=black
}

% -- No Header dan No Footer ---
\pagestyle{plain}

% --- Ubah nama daftar listing ke "DAFTAR KODE" ---
% --- Harus diletakkan sebelum \begin{document} ---
\renewcommand{\lstlistlistingname}{\centering\normalsize DAFTAR KODE} 
\renewcommand{\lstlistingname}{Kode}
\lstset{basicstyle=\ttfamily\footnotesize,breaklines=true}
%\captionsetup[lstlisting]{justification=raggedright,singlelinecheck=false}


\renewcommand \cftchapdotsep{4.5}
%\cftsetrmarg{4em}  % Ini diset agar judul subbab tidak mepet ke nomor halaman di daftar isi
% -- Atur indentasi judul bab, subbab, dan subsubbab di daftar isi ---
\setlength{\cftchapindent}{0em}
\setlength{\cftsecindent}{0em}
\setlength{\cftsubsecindent}{0em}
\setlength{\cftsubsubsecindent}{0em}

% --- Ubah nama bulan ke Bahasa Indonesia ---
\renewcommand*{\DTMbahasaimonthname}[1]{%
  \ifcase#1 %
    \or Januari%  % 1st month
    \or Februari%  % 2nd month
    \or Maret%  % 3rd month
    \or April%  % 4th month
    \or Mei%  % 5th month
    \or Juni%  % 6th month
    \or Juli%  % 7th month
    \or Agustus%  % 8th month
    \or September%  % 9th month
    \or Oktober%  % 10th month
    \or November%  % 11th month
    \or Desember%  % 12th month
  \fi
}

% ==========================================
% AWAL DOKUMEN
% ==========================================
\begin{document}

% ==========================================
% HALAMAN JUDUL
% ==========================================
\begin{titlepage}
\begin{center}

    
    \vspace*{2cm}
    
    {\Large\bfseries PERANCANGAN SISTEM INFORMASI AKADEMIK BERBASIS WEB}\\
     \vspace{4cm}

    {\Large \textbf{Proposal Tugas Akhir}}\\


    \vspace{2cm}
    
    
    {\large Oleh}\\[0.3cm]
    \textbf{
    {\large John Doe}\\
    {\large 18299000}
    }\\

    \vspace{2cm}
    
    \begin{figure}[h]
    \centering
    \includegraphics[width=0.2\textwidth]{image/ganesha.jpg}
    \end{figure}
    
    
    % \vspace{1cm}
    \vfill

    \textbf{
    {\large PROGRAM STUDI SISTEM DAN TEKNOLOGI INFORMASI}\\
    {\large SEKOLAH TEKNIK ELEKTRO DAN INFORMATIKA}\\
    {\large INSTITUT TEKNOLOGI BANDUNG}\\
    {\large \DTMlangsetup{showdayofmonth=false,showmonthname=true,showyear=true}\today}
    }
\end{center}
\end{titlepage}



% ==========================================
% LEMBAR PENGESAHAN 
% ==========================================
\newpage
\thispagestyle{empty}
\pagenumbering{gobble}
\begin{center}
  \textbf{\large LEMBAR PENGESAHAN}\\[1cm]
  \vspace*{1.5cm}
    
  {\large\bfseries PERANCANGAN SISTEM INFORMASI AKADEMIK BERBASIS WEB}\\
     \vspace{2cm}

  {\Large \textbf{Proposal Tugas Akhir}}\\


  \vspace{1.5cm}
    
    
  {\large Oleh}\\[0.3cm]
    \textbf{
    {\large John Doe}\\
    {\large 18299000}
  }\\
    
  \vspace{0.5cm}
 
  {\large Program Studi Sistem dan Teknologi Informasi}\\
  {\large Sekolah Teknik Elektro dan Informatika}\\
  {\large Institut Teknologi Bandung}\\

  \vspace{1.5cm}

  Proposal Tugas Akhir ini telah disetujui dan disahkan\\ 
  di Bandung, pada tanggal \today\\[1cm]

% ==========================================
% Versi 1 pembimbing (default)
% ==========================================
	Pembimbing  \\[3cm]
	Dr. Ir. John Doe, M.T.   \\[0.2cm]
	NIP. 123456789 
% ==========================================

\end{center}

\vspace{1cm}
\noindent

% ==========================================
% Jika ada 2 pembimbing TA, uncomment dan edit 
% tabular di bawah ini. Kemudian, comment out atau hapus
% bagian versi 1 pembimbing di atas.
% ==========================================

%\begin{tabular}{p{1cm}p{7cm}p{7cm}}
%   & Pembimbing 1 & Pembimbing 2 \\[3cm]
%   & Dr. Ir. John Doe, M.T. & Dr. Mary Doe, M.Sc. \\[0.2cm]
%   &  NIP. 123456789 & NIP. 987654321
%\end{tabular}



% -- Change page number style to roman ---
\pagenumbering{roman} 


% ==========================================
% DAFTAR ISI, TABEL, GAMBAR
% ==========================================
% --- DAFTAR ISI ---
\makeatletter
\renewcommand{\tableofcontents}{%
  \clearpage
  \thispagestyle{plain}% no header
  \begin{center}
    {\large\bfseries\MakeUppercase{\contentsname}\par}
  \end{center}
  \vskip 1em
  \@starttoc{toc}%
}
\makeatother

\newpage
\renewcommand{\cfttoctitlefont}{\hfill\large\bfseries\MakeUppercase}
\renewcommand{\cftaftertoctitle}{\hfill}
\tableofcontents
%\addcontentsline{toc}{chapter}{DAFTAR ISI}

% --- DAFTAR GAMBAR ---
\newpage
\renewcommand{\cftloftitlefont}{\hfill\large\bfseries\MakeUppercase}
\renewcommand{\cftafterloftitle}{\hfill}
\listoffigures
\addcontentsline{toc}{chapter}{DAFTAR GAMBAR}

% --- DAFTAR TABEL ---
\newpage
\renewcommand{\cftlottitlefont}{\hfill\large\bfseries\MakeUppercase}
\renewcommand{\cftafterlottitle}{\hfill}
\listoftables
\addcontentsline{toc}{chapter}{DAFTAR TABEL}

% --- DAFTAR LISTING (ALGORITMA, PSEUDOCODE, SOURCE CODE) ---
\newpage

\lstlistoflistings
\addcontentsline{toc}{chapter}{\lstlistlistingname}

\mainmatter
% --- FORMAT TAMPILAN JUDUL BAB, SUBBAB, JUDUL GAMBAR DAN TABEL ---
% --- Judul Bab ---
\titleformat{\chapter}[display]
      {\centering\normalfont\large\bfseries} % Commands for the entire chapter title
      {\MakeUppercase \chaptertitlename\ \thechapter}{0pt}{\large} % Chapter number format
\renewcommand\thechapter{\Roman{chapter}}
% --- Judul Subbab dan Subsubbab ---
\titleformat{\section}
	{\normalfont\bfseries}
	{\thesection}{1em}{}
\titleformat{\subsection}
	{\normalfont\bfseries}
	{\thesubsection}{1em}{}
\titleformat{\subsubsection}
	{\normalfont\bfseries}
	{\thesubsubsection}{1em}{}


% --- Format judul gambar dan tabel ---
\captionsetup[figure]{labelsep=space}
\captionsetup[table]{labelsep=space}
\floatsetup[table]{capposition=top}
\captionsetup[lstlisting]{labelsep=space}
\floatsetup[lstlisting]{capposition=top}

% --- Atur indentasi paragraf ---
\setlength{\parindent}{0pt}
% -- Change page number style to arabic ---
\pagenumbering{arabic} 

% ==========================================
% BAB I PENDAHULUAN
% ==========================================
\chapter{PENDAHULUAN}
\label{chap:pendahuluan}
% --- Latar Belakang ---
\section{Latar Belakang}
Dalam pengolaan gedung cerdas, aspek keamanan dan keselamatan dalam gedung yang terintegrasi merupakan salah satu aspek yang sangat penting untuk segera diimplementasikan dengan baik. Aktivitas perusahaan, karyawan maupun pengunjung yang beragam dan kompleks membutuhkan adanya suatu sistem kontrol akses yang bukan hanya mengatur keluar masuknya karyawan dan pengunjung, tetapi juga dapat mendukung dan membantu saat terjadinya keadaan darurat atau bencana. Sistem yang dapat membantu mengontrol dan mengenali karyawan atau pengunjung secara tepat dan cepat menjadi salah satu elemen penting dalam mewujudkan hal tersebut. 

Kebutuhan sistem ini juga disebutkan dalam Peraturan Menteri Pekerjaan Umum dan Perumahan Rakyat (PUPR) Nomor 10 Tahun 2023 tentang Bangunan Gedung Cerdas. Regulasi ini menyebutkan pada Pasal 4 ayat (2) bahwa prinsip Bangunan Gedung Cerdas (BGC) harus memiliki sistem-sistem yang bekerja otomatis dan terintegrasi satu sama lain. Peraturan ini juga menyebutkan pada Pasal 5 bahwa elemen BGC harus terintegrasi dalam Sistem Manajemen Bangunan Gedung (\textit{building management system}), dimana elemen terebut diantaranya adalah sistem kontrol akses gedung. 

Science Techno Park (STP) Gedebage, atau biasa disebut juga ITB Innovation Park (IIP) Bandung Technopolis, merupakan infrastruktur yang dibangun untuk mendorong inovasi dan komersialisasi produk-produk teknologi milik Institut Teknologi Bandung (ITB). Saat ini, Gedung IIP masih belum menerapkan sistem kontrol akses yang otomatis dapat diintegrasikan dengan sistem-sistem lain yang ada pada gedung. Hal ini dapat menyebabkan inefisiensi dalam pengelolaan akses karyawan yang penghuni yang diestimasi dapat menampung hingga 1200 orang. Gedung IIP membutuhkan membutuhkan sistem kontrol akses yang mampu memverifikasi identitas karyawan secara cepat tanpa memyebabkan antrian, serta memberikan karyawan kemudahan dalam menggunakannya. Namun, penerapan kontrol akses otomatis tersebut tetap harus dilakukan dengan pertimbangan keselamatan penghuni dan karyawan gedung saat keadaan darurat dan bencana.

Kebutuhan akan sistem yang mempertimbangkan keselamatan dibahas dalam Peraturan Pemerintah (PP) Nomor 16 Tahun 2021 tentang Peraturan Pelaksanaan Undang-Undang Nomor 28 Tahun 2002 tentang Bangunan Gedung. Peraturan ini menyebutkan pada Pasal 28 ayat (1) bahwa setiap Bangunan Gedung harus memenuhi ketentuan aspek keselamatan Bangunan gedung. Berdasarkan peraturan ini, dapat disimpulkan bahwa pengembangan fasilitas gedung harus dilakukan dengan keselamatan sebagai pertimbangan. Aspek keselamatan juga dibahas pada Surat Edaran Menteri Pekerjaan Umum Nomor 22/SE/M/2024 tentang Pedoman Penilaian Kinerja Bangunan Gedung Cerdas Tahap Pemanfaatan Dan Pemeriksaan Kinerja Bangunan Gedung Cerdas Tahap Pembongkaran. Surat edaran ini menyebutkan bahwa sistem kontrol akses membutuhkan panduan operasional terkait titik kumpul yang berisi prosedur evakuasi dan pengumpulan titik aman dalam keaadaan darurat. 

Teknologi gerbang \textit{fail-safe} otomatis dengan pengenalan wajah dan mekanisme darurat hadir sebagai solusi yang relevan untuk menjawab kebutuhan tersebut. Dibandingkan metode biometrik lain seperti sidik jari yang memerlukan kontak fisik, pengenalan wajah menawarkan mekanisme tanpa kontak yang lebih higenis dan cepat \autocite{Gupta}. Penerapan teknologi ini memungkinkan adanya sistem kontrol akses yang efisien tanpa hambatan fisik yang berarti. Penggunaan gerbang yang terbuka dalam kondisi tanpa listrik (\textit{fail-safe}) dan mekanisme darurat sangat relevan sebagai langkah untuk mencegah sistem kontrol akses menjadi hambatan dalam penerapan keselamatan Bangunan Gedung saat keadaan darurat.  

Berdasarkan analisis tersebut, tugas akhir ini mengusulkan perancangan sistem kontrol akses berbasis pengenalan wajah untuk kondisi darurat dan bencana pada bangunan cerdas. Sistem ini dirancang sebagai solusi menyeluruh yang meliputi perangkat keras dan perangkat lunak sistem. Sistem akan diwujudkan dalam bentuk prototipe fungsional yang menintegrasikan komponen-komponen tertentu yang dapat berperan sebagai kontrol akses pada gedung dengan tetap mempertimbangkan aspek keselamatan.
% --- Rumusan Masalah ---
\section{Rumusan Masalah}
Saat ini, Gedung IIP belum memiliki sistem kontrol akses yang bekerja secara otomatis dan terintegrasi dalam Sistem Manajemen Bangunan Gedung. Oleh sebab itu, diperlukan pengembangan sistem kontrol akses yang dapat menjadi solusi sebagai kontrol akses gedung yang tetap mempertimbangkan aspek keselamatan.

Masalah yang terjadi dapat dirumuskan sebagai berikut : 
\begin{enumerate}
\item	Bagaimana merancang sistem kontrol akses yang dapat terintegrasi dengan \textit{Building Management System} yang dimiliki gedung?
\item	Bagaimana merancang sistem kontrol akses yang tetap mempertimbangkan aspek keselamatan pada gedung dalam keaadaan darurat?
\end{enumerate}
Untuk dapat memenuhi persyaratan gedung cerdas serta meningkatkan keamanan dan keselamatan, gedung IIP membutuhkan sistem gerbang otomatis sebagai kontrol akses yang mampu mengatur dan mengetahui setiap pengunjung gedung, serta memiliki aspek keselamatan yang mumpuni dalam keadaan darurat. Sistem juga harus terintegrasi dengan \textit{Building Management System} yang dimiliki gedung sebagaimana diatur dalam Peraturan Menteri PUPR.     

% --- Tujuan ---
\section{Tujuan}
Tujuan dari pelaksanaan tugas akhir ini adalah untuk merancang dan mengembangkan sebuah sistem kontrol akses berupa gerbang otomatis yang mampu terintegrasi dengan \textit{Building Management System} yang dimiliki gedung. Sistem ini juga diharapkan dapat dirancang sehingga dapat membantu sistem keselamatan yang ada pada gedung ketika terjadi keadaan darurat.
Kriteria keberhasilan tugas akhir ini meliputi : 
\begin{enumerate}
\item	Sistem mampu terintegrasi dengan \textit{Building Management System} yang dimiliki gedung dengan memberikan informasi tentang orang-orang yang telah memasuki gedung melalui gerbang.
\item   Sistem dapat memenuhi kebutuhan gedung dengan tetap memperhatikan aspek keselamatan gedung saat terjadi keadaan darurat dengan menetralisir hambatan yang mungkin terjadi pada gerbang.
\end{enumerate}   

% --- Batasan Masalah ---
\section{Batasan Masalah}
Untuk memastikan pengembangan sistem terarah dan sejalan dengan kebutuhan, dirumuskan batasan masalah sebagai pedoman dalam pelaksanaan tugas akhir ini. batasan masalah tersebut meliputi:
\begin{enumerate}
    \item Pengguna sistem yang dilibatkan adalah pihak pengelola gedung IIP beserta salah satu perusahaan yang menggunakan gedung IIP.
    \item Sistem yang dikembangkan hanya mencakup satu unit gerbang sesuai ketersediaan sumber daya, namun dirancang dan dikembangkan sebagai representasi dari keseluruhan sistem.
    \item Sistem akan dikembangkan menggunakan basis data independen yang tidak terintegrasi langsung dengan data yang dimiliki gedung. 
    \item Pengujian sistem dilakukan dalam skala kecil pada lingkungan atau area terbatas di Gedung IIP.
\end{enumerate}
% --- Metodologi Pengerjaan TA ---
\section{Metodologi}
Metodologi yang digunakan pada pengerjaan tugas akhir ini didasarkan pada metodologi \textit{design thinking}. \textit{Design thinking} adalah proses iteratif yang berorientasi pada manusia (\textit{human-centered approach}) dan kolaborasi antara pengembang sistem dan penggunanya. Metodologi ini menghadirkan solusi inovatif berdasarkan bagaimana pengguna berfikir, bertindak, dan merasakan sesuatu. 
\textit{Design thinking} memiliki lima tahapan utama, yaitu \textit{Empathize}, \textit{Define}, \textit{Ideate}, \textit{Prototype}, dan \textit{Test}.
\begin{enumerate}
\item \textit{Empathize} \newline
Tahap ini bertujuan untuk memahami pengguna, baik permasalahan yang dialami, pemahaman dan pengalaman yang mereka miliki di dalam konteks desain sistem. Dalam melakukan \textit{empathize}, pengembang dapat mengumpulkan data secara aktif dengan melakukan \textit{observation} (\textit{observe}), wawancara, survei, dan percakapan langsung (\textit{engage}), ataupun kombinasi antara \textit{observe} dan \textit{engage}. 
\item \textit{Define} \newline
Tahap ini menggunakan hasil yang didapatkan pada tahap \textit{empathize} untuk menemukan dan merumuskan masalah pengguna. Data yang dikumpulkan dapat diolah menjadi pemahaman atau \textit{insight} dan digunakan untuk membentuk \textit{problem statement} atau \textit{point-of-view}. \textit{Problem statement} berperan menjadi petunjuk dalam membuat sistem yang menyelesaikan masalah pengguna. Rumusan masalah yang baik harus spesifik, terukur, dan mampu memicu inspirasi dan kreativitas dalam pencarian solusi.
\item \textit{Ideate} \newline
Tahap ini berfokus pada eksplorasi ide dan solusi yang dapat digunakan sebagai solusi dalam menyelesaikan masalah yang dirumuskan. Pengembang dapat melakukan beragam cara seperti \textit{brainstorming}, \textit{mindmapping}, \textit{sketching}, hingga bahkan \textit{prototyping} untuk mendapatkan solusi-solusi potensial dalam menyelesaikan masalah pengguna.
\item \textit{Prototype} \newline
Tahap ini merupakan proses mewujudkan ide menjadi bentuk nyata yang dapat digunakan untuk mendapatkan jawaban tentang apakah ide yang telah dipilih dapat menyelesaikan permasalahan yang dialami pengguna. \textit{Prototype} yang dihasilkan dapat berupa model fisik, simulasi, sketsa, ataupun wujud lain yang dapat merepresentasikan ide yang dipilih. Tujuan utama dari tahapan ini bukanlah membuat produk akhir, melainkan sebagai langkah dalam mencoba berbagai macam ide dan kemungkinan untuk mendapatkan data dan pemahaman lebih lanjut mengenai permasalahan dan solusi. 
\item \textit{Test} \newline
Tahapan ini bertujuan untuk mendapatkan umpan balik menggunakan \textit{prototype} yang telah dihasilkan untuk mendapatkan informasi dan data baru dari pengguna sistem yang akan dibuat. Tahap \textit{test} berperan dalam menyempurnakan \textit{prototype} dan solusi, mengenali pengguna lebih lanjut, hingga menyempurnakan \textit{point-of-view} atau \textit{problem statement}.
\end{enumerate}
% ==========================================
% BAB II STUDI LITERATUR
% ==========================================
\chapter{STUDI LITERATUR}
\label{chap:studi-literatur}
\section{Regulasi Bangunan Gedung Cerdas}
Perancangan sistem kontrol akses pada gedung cerdas tentunya memiliki regulasi dan standar yang berlaku. Di Indonesia, \textcite{PermenPUPR_10_2023} telah menetapkan standar bangunan cerdas dalam Peraturan Menteri Pekerjaan Umum dan Perumahan Rakyat Nomor 10 Tahun 2023 tentang Bangunan Gedung Cerdas, dimana pada Pasal 5 ayat (3) disebutkan bahwa sistem kontrol akses merupakan salah satu elemen bangunan gedung cerdas (BGC) yang harus menggunakan teknologi tinggi dan terintegrasi untuk mewujudkan BGC. Regulasi ini juga digunakan oleh \textcite{mpu22_2024} dalam Surat Edaran Menteri Pekerjaan Umum Nomor 22/SE/M/2024, yang merupakan landasan utama dalam penilaian kinerja bangunan gedung cerdas.  
\section{Regulasi tentang keselamatan Bangunan Gedung}
Regulasi tentang keselamatan Bangunan Gedung dibahas dalam Peraturan Pemerintah Nomor 16 Tahun 2021 tentang Peraturan Pelaksanaan Undang-Undang Nomor 28 Tahun 2002 tentang Bangunan Gedung. Peraturan ini menyebutkan pada Pasal 28 ayat (1) bahwa setiap Bangunan Gedung harus memenuhi ketentuan aspek keselamatan gedung, salah satunya adalah keadaan darurat berupa kebakaran.

Aspek keselamatan bangunan gedung juga disebutkan pada salah satu kriteria penilaian sistem pada Bangunan Gedung Cerdas. Surat Edaran Menteri Pekerjaan Umum Nomor 22/SE/M/2024 menyebutkan bahwa salah satu Kinerja Unjuk Kerja (KUK) untuk kontrol akses adalah panduan operasional terkait titik kumpul yang penting berisi prosedur evakuasi dan pengumpulan titik aman dalam keadaan darurat, yang merupakan salah satu aspek keselamatan dalam gedung.
\section{Sistem Kontrol Akses}
Sistem kontrol akses adalah metode otomatis yang mengizinkan pihak yang diasumsikan teman untuk memasuki area yang dikontrol, dibatasi atau diamankan dengan penyaringan di portal akses kontrol yang disediakan. Sistem kontrol akses dirancang untuk memastikan bahwa hanya orang yang berwenang atau memenuhi syarat saja yang diizinkan untuk memasuki area ekslusif tersebut \autocite{norman2011electronic}. Lebih lanjut, \textcite{norman2011electronic} menyebut tentang Sistem kontrol akses elektronik yang memanfaatkan komputer, kredensial, pembaca kredensial, dan pintu kunci untuk mengontrol akses secara elektronik. Elemen pada pintu juga termasuk pada alarm dan sensor keluar yang digunakan untuk keadaan tertentu.
\section{Gerbang pada Bangunan Gedung sebagai Kontrol Akses}
Terdapat beberapa ketentuan dan rekomendasi tentang bagaimana spesifikasi dari gerbang yang digunakan sebagai kontrol akses dalam bangunan gedung. \textcite{pup_r14_2017} menyebutkan bahwa arah bukaan pintu terpasang pada ruang yang digunakan oleh pengguna dalam jumlah besar harus terbuka searah dengan arah ke luar Bangunan Gedung/ruang. Kemudian dalam lampiran teknisnya, peraturan ini juga menyebutkan bahwa pintu akses (turnstile) memiliki lebar efektif bukaan paling sedikit 60 cm dan untuk disabilitas pintu harus memilik lebar efektif paling sedikit 80 cm. Peraturan ini dapat kita adopsi sebagai referensi menentukan dimensi gerbang yang digunakan pada sistem kontrol akses. Gambar \ref{gambar:contoh-pintu-akses} menunjukkan contoh penerapan pada pintu akses pada lampiran Peraturan Menteri Pekerjaan Umum dan Perumahan Rakyat Nomor 14/PRT/M/2017 Tahun 2017 tentang Persyaratan Kemudahan Bangunan Gedung.
\begin{figure}[H] % pilihan opsi yang disarankan: t = top, b = bottom, h = here
	\centering
  \captionsetup{justification=centering}
    	\includegraphics[width=0.7\textwidth]{image/pintu-akses-2.png}
	\caption{Contoh penerapan desain pada pintu akses (\textit{turnstile})}
	\label{gambar:contoh-pintu-akses}
\end{figure}

Sebagai rekomendasi, \textcite{ding2021winggate} menyebutkan bahwa gerbang dengan bentuk \textit{wedge} lebih baik digunakan dibandingkan dengan gerbang berbentuk flat. Pada jurnal tersebut, hasil eksperimen menunjukkan bahwa gerbang bertipe \textit{wedge} memiliki kontrol evakuasi dan keramaian yang lebih baik dibandingkan dengan gerbang bertipe \textit{flat}.

\section{Sistem Pengenalan Wajah}
Pengenalan wajah adalah teknologi dalam visi komputer yang digunakan untuk mengidentifikasi seseorang atau suatu objek dari gambar atau video. Pengenalan wajah adalah masalah pengenalan pola visual, dimana wajah sebagai objek tiga dimensi sebagai subjek yang memiliki pencahayaan, pose, dan ekspresi yang bervariasi untuk diidentifikasi berdasarkan gambar dua dimensi yang diambil \autocite{li2024handbook_face_recognition}. Berdasarkan \textcite{li2024handbook_face_recognition} dalam bukunya \citetitle{li2024handbook_face_recognition}, sistem pengenalan wajah terdiri atas empat modul utama, yaitu \textit{detection}, \textit{aligment}, \textit{feature extraction}, dan \textit{matching}, dimana lokalisasi dan normalisasi (tahapan \textit{face detection} dan \textit{aligment}) adalah langkah proses yang dilakukan sebelum pengenalan wajah (\textit{facial feature extraction} dan \textit{matching}).

Pengenalan wajah sudah menjadi biometrik untuk melakukan autentikasi yang digunakan secara luas di berbagai bidang, seperti militer, keuangan, keamanan publik, hingga kehidupan sehari hari. \textcite{Jadhav_2024} menyebutkan bahwa berdasarkan survey dari HID Global pada tahun 2024, jumlah bisnis yang menggunakan biometrik sebagai kontrol akses mereka naik dari 30 persen menjadi 39 persen pada dua tahun terakhir, yang menunjukkan bahwa penggunaan biometrik, termasuk pengenalan wajah sudah mulai diadopsi secara cepat oleh pelaku bisnis.
\subsection{Algoritma Pengenalan Wajah}
Algoritma pengenalan wajah berbasis \textit{Deep Learning} bekerja dengan mengekstrak ciri unik wajah menjadi vektor fitur. Schroff dkk. (2015) memperkenalkan arsitektur FaceNet yang memetakan citra wajah ke dalam ruang Euclidean menggunakan fungsi kerugian \textit{triplet loss}, di mana jarak antar titik merepresentasikan kemiripan wajah \parencite{schroff2015facenet}.

Namun, perkembangan terkini menawarkan pendekatan fungsi kerugian berbasis margin sudut (\textit{angular margin}) yang lebih diskriminatif. Deng dkk. (2018) mengusulkan ArcFace (\textit{Additive Angular Margin Loss}) yang terbukti meningkatkan akurasi secara signifikan dengan memperjelas batasan antar kelas identitas dalam ruang fitur \parencite{deng2018arcface}. Selain itu, Meng dkk. (2021) mengembangkan MagFace, sebuah metode yang mengoptimalkan besaran (\textit{magnitude}) vektor fitur agar berkorelasi dengan kualitas wajah, sehingga sistem dapat mengenali sekaligus menilai kualitas citra input secara intrinsik \parencite{meng2021magface}.

Dari sisi arsitektur jaringan (\textit{backbone}), implementasi pada perangkat \textit{edge} dengan sumber daya terbatas memerlukan model yang efisien. Chen dkk. (2018) merancang MobileFaceNets, sebuah arsitektur CNN ringan yang menggunakan \textit{depth-wise separable convolution}. Model ini mampu mereduksi beban komputasi secara drastis dibandingkan model standar seperti ResNet-50, namun tetap mempertahankan akurasi tinggi untuk verifikasi wajah secara \textit{real-time} pada perangkat seluler atau IoT \parencite{chen2018mobilefacenets}.

Selain algoritma pengenalan, skalabilitas pencarian data juga menjadi perhatian. Malkov dan Yashunin (2018) mengusulkan penggunaan metode \textit{Hierarchical Navigable Small World} (HNSW) untuk pencarian tetangga terdekat yang efisien pada data vektor dimensi tinggi. Teknik ini memastikan waktu respons sistem tetap stabil dan cepat meskipun basis data pengguna terus bertambah \parencite{malkov2018hnsw}.

% Teknologi pengenalan wajah telah mengalami perkembangan yang progresif dimana saat ini telah berkembang pendekatan \textit{Deep Learning} yang menawarkan akurasi pengenalan yang tinggi. metode ini telah dibahas oleh \textcite{Gupta} yang menunjukkan bahwa penggunaan pustaka modern seperti OpenCV dengan model \textit{Deep Learning} mampu melakukan deteksi dan pengenalan wajah secara \textit{real-time} dengan tingkat kesalahan rendah, bahkan dalam kondisi lingkungan yang dinamis. 
\section{Parameter Evaluasi Kinerja Biometrik}
Untuk mengukur keandalan sistem pengenalan wajah, standar internasional ISO/IEC 19795-1 menetapkan beberapa metrik utama yang digunakan dalam pengujian \parencite{iso19795}:
\begin{enumerate}
    \item Akurasi atau \textit{Accuracy}, yaitu rasio total prediksi yang benar terhadap total seluruh percobaan.
    \item \textit{False Acceptance Rate} (FAR), yaitu tingkat kesalahan sistem dalam mengenali orang asing atau tidak terdaftar sebagai pengguna yang sah. Dalam konteks keamanan gedung, nilai FAR harus ditekan serendah mungkin.
    \item \textit{False Rejection Rate} (FRR), yaitu tingkat kesalahan sistem dalam menolak pengguna yang sebenarnya sudah terdaftar dan memiliki hak akses. Nilai FRR yang tinggi dapat mengganggu kenyamanan pengguna.
    \item Waktu Respons atau \textit{Latency}, yaitu waktu yang dibutuhkan sistem mulai dari wajah terdeteksi kamera hingga sinyal kontrol dikirimkan ke aktuator.
\end{enumerate}
\section{Keamanan Data dan Privasi}
Sistem pengenalan wajah atau biometrik memiliki beberapa landasan hukum di Indonesia, terutama terkait dengan beberapa aturan tentang keamanan dan privasi data yang akan dijelaskan sebagai berikut.

\input table/tabelbab2-1.tex

Berdasarkan studi terbaru di Indonesia, beberapa isu hukum dan etika yang muncul bila menerapkan pengenalan wajah diantaranya adalah Persetujuan dan Kesadaran Subjek Data, Privasi dan Pengamanan Data, Kesalahan Identifikasi dan Risiko Penyalahgunaan, hingga Transparansi dan Akuntabilitas.

\section{Integrasi Sistem}
Integrasi antara sistem perangkat lunak cerdas dan aktuator fisik atau gerbang umumnya dilakukan melalui mekanisme kontak kering atau \textit{dry contact} menggunakan modul relay. Kainz dkk. (2019) mendemonstrasikan skema di mana pin GPIO pada mikrokontroler atau minikomputer digunakan untuk memicu relay, yang kemudian bertindak sebagai saklar elektronik untuk sirkuit pengunci pintu \parencite{kainz2019raspberry}.

Selain itu, Wu dkk. (2020) dalam studinya mengenai sistem kontrol akses menekankan pentingnya integrasi sensor keamanan untuk mencegah gerbang menutup saat pengguna masih berada di jalur lintasan. Prinsip-prinsip ini diadopsi dalam perancangan untuk memastikan sistem tidak hanya aman secara digital, tetapi juga aman secara fisik bagi pengguna \parencite{wu2020design}.

\section{\textit{Design Thinking}}
Metodologi \textit{Design Thinking} adalah sebuah kerangka kerja iteratif yang berfokus pada pemahaman mendalam terhadap pengguna sebagai dasar pengembangan solusi.
\textcite{Plattner_2010} menyatakan bahwa proses ini dimulai dari tahap \textit{Empathize}, yaitu pengumpulan wawasan mengenai perilaku, kebutuhan, dan tantangan pengguna melalui observasi maupun interaksi langsung.
Selanjutnya, tahap \textit{Define} digunakan untuk menyusun dan merumuskan inti permasalahan berdasarkan temuan yang telah terkumpul.
Pada tahap \textit{Ideate}, berbagai kemungkinan solusi dieksplorasi dan dikembangkan melalui teknik kreatif seperti \textit{brainstorming} atau pemetaan ide.
Tahap berikutnya adalah \textit{Prototype}, yaitu pembuatan representasi awal dari solusi agar dapat diuji dan divalidasi secara cepat.
Terakhir, tahap \textit{Test} dilakukan untuk mengumpulkan umpan balik pengguna, yang kemudian menjadi dasar untuk penyempurnaan solusi secara berulang \autocite{Plattner_2010}.
% \section{Pengembangan Sistem Gerbang Otomatis pada KAI}
% Layanan \textit{Face Recognition Boarding} KAI merupakan inovasi digital yang diluncurkan oleh PT Kereta Api Indonesia (KAI) untuk mempermudah proses keberangkatan penumpang kereta api jarak jauh. Melalui sistem ini, penumpang tidak lagi perlu menunjukkan kartu identitas (KTP) atau mencetak \textit{boarding pass}, karena proses verifikasi identitas dilakukan secara otomatis menggunakan teknologi pengenalan wajah. Teknologi ini mulai diterapkan di beberapa stasiun besar seperti Stasiun Gambir sejak 1 September 2023, dan akan diperluas secara bertahap ke stasiun-stasiun lainnya di Indonesia.

% Untuk dapat menggunakan layanan ini, penumpang diwajibkan melakukan pendaftaran awal melalui aplikasi Access by KAI (sebelumnya dikenal sebagai KAI Access). Proses pendaftaran meliputi unggah data diri berupa Nomor Induk Kependudukan (NIK) sesuai KTP dan foto wajah atau swafoto yang jelas. Data tersebut kemudian akan diverifikasi dan disinkronkan dengan sistem tiket elektronik KAI. Setelah verifikasi berhasil, data wajah penumpang akan tersimpan dalam sistem, dan penumpang dapat langsung melakukan \textit{boarding} hanya dengan menatap kamera di gate yang sudah dilengkapi sensor pengenal wajah.

% Pada saat keberangkatan, sistem akan mencocokkan wajah penumpang dengan data tiket dan identitas yang telah terdaftar. Jika data valid dan sesuai, gerbang otomatis terbuka, memungkinkan penumpang masuk tanpa perlu melakukan antri atau pemeriksaan manual. Teknologi ini bertujuan untuk meningkatkan efisiensi dan keamanan proses \textit{boarding}, mengurangi risiko pemalsuan identitas atau penyalahgunaan tiket, serta memberikan pengalaman perjalanan yang lebih modern dan cepat. Namun, KAI sendiri tidak menjadikan layanan \textit{Face Recognition Boarding} KAI sebagai alur utama dalam memasuki kereta. Untuk menjadikan Sistem Gerbang dengan pengenalan wajah sebagai alur utama, sistem tetap harus memberikan alternatif alur akses sebagai langkah penegakan UU No. 27 Tahun 2022 tentang Perlindungan Data Pribadi (UU PDP).



% \section{Kesenjangan}
% Penelitian selama 5 tahun terakhir menunjukkan bahwa belum adanya suatu mekanisme ataupun ketentuan implementasi yang harus dilakukan untuk mengembangkan sistem kontrol akses yang mempertimbangkan keselamatan secara eksplisit. Hal ini membuat diperlukannya penelitian lebih lanjut secara mandiri tentang pengembangan sistem gerbang otomatis berbasis pengenalan wajah untuk kondisi darurat dan bencana. 
% ============================================================================================
% BAB III ANALISIS MASALAH
% Pembagian subbab tidak rigid dan dapat bervariasi. Bab ini minimal berisi analisis kebutuhan
% fungsional dan nonfungsional, analisis berbagai alternatif solusi yang dapat ditawarkan, dan
% metode pemilihan solusi yang diusulkan.
% ============================================================================================
\chapter{ANALISIS MASALAH}
\label{chap:analisis-masalah}
\section{Analisis Kondisi Saat Ini}
Menurut \textcite{laudon2020}, gambarkan terlebih dahulu model konseptual sistem yang ada saat ini. Model konseptual ini berisi berbagai komponen atau subsitem dan interaksi antarsubsistem tersebut. Setelah itu, berikan penjelasan tentang masalah yang ada pada sistem tersebut. Paragraf berikut berisi contoh penjabaran masalah sistem informasi fasilitas kesehatan untuk pasien \autocite{pressman2019}. 
\section{Analisis Kebutuhan}
\lipsum[4]
\subsection{Identifikasi Masalah Pengguna}
\lipsum[5]
\subsection{Kebutuhan Fungsional}
\lipsum[6]
\subsection{Kebutuhan Nonfungsional}
\lipsum[7]

\section{Analisis Pemilihan Solusi}
\subsection{Alternatif Solusi}
\lipsum[8]
\subsection{Analisis Penentuan Solusi}
\lipsum[9]
% ==========================================
% BAB IV DESAIN KONSEP SOLUSI
% ==========================================
\chapter{DESAIN KONSEP SOLUSI}
\label{chap:desain-konsep-solusi}
Fokus utama pada desain konsep solusi adalah menjelaskan model konseptual dan penjelasan desain yang dipilih pada bab sebelumnya mengenai pengenalan wajah untuk kontrol akses di lobi ITB Innovation Park.
\section{Diagram Konseptual}
Penerapan desain solusi yang dipilih akan mempengaruhi alur kontrol akses yang telah ada sebelumnya. Gambar X adalah diagram alur kontrol akses sebelum penerapan sistem dan gambar y merupakan alur kontrol akses sesudah penerapan sistem pada Gedung IIP.

Pada kedua gambar, dapat diliihat dengan jelas peruubahan alur yang terjadi, dimana setelah melewati petugas keamanan luar gedung, pengguna harus melewati sistem gerbang terlebih dahulu sebelum akhirnya dapat mengakses lift pada lobi gedung.

\section{Penjelasan Desain}

% ==========================================
% BAB V RENCANA SELANJUTNYA
% ==========================================
\chapter{RENCANA SELANJUTNYA}
\label{chap:rencana-selanjutnya}
Bab ini menjelaskan langkah implementasi sistem yang akan dilakukan kedepannya. Rencana ini dijabarkan dalam bentuk linimasa pekerjaan, yang kemudian dilanjutkan dengan rencana implementasi, desain pengujian, serta analisis risiko. 
\section{Linimasa Pengerjaan}
\section{Rencana Implementasi}
\section{Rencana Evaluasi}
\section{Analisis Risiko}


\backmatter

% ==========================================
% DAFTAR PUSTAKA
% ==========================================

\printbibliography[title={DAFTAR PUSTAKA}]

% ==========================================
% LAMPIRAN (optional)
% ==========================================
\appendix
% Uncomment baris di bawah ini jika ada lampiran
% \chapter{LAMPIRAN A. HASIL WAWANCARA PENGGUNA-1}


% \section{Perangkat Lunak untuk Akuisisi Data dari Sensor Ultrasonik}
% \lstinputlisting[language=Python, caption=source code untuk akuisisi data dari sensor ultrasonik]{code/ultrasonic_data_acquisition.py}  
% \section{Perangkat Lunak untuk Akuisisi Data dari Sensor Suhu dan Kelembaban}
% \lstinputlisting[language=Python, caption=source code untuk akuisisi data dari sensor suhu dan kelembaban]{code/temp_humidity_data_acquisition.py}  

% \chapter{LAMPIRAN B. HASIL WAWANCARA PENGGUNA-2}

\end{document}
 